\documentclass[12pt,letterpaper]{article}
%\documentstyle[11pt]{article}
\usepackage[utf8]{inputenc}
\usepackage{amsmath}
\usepackage{xfrac}
\usepackage{amsfonts}
\usepackage{amssymb}
\usepackage[version = 3]{mhchem}
\usepackage{chemstyle}
%%For Table perhaps%%
%\usepackage{graphics}
\usepackage{graphicx}
\usepackage{epstopdf}
%\usepackage{tabularx,ragged2e,booktabs,caption}
%\newcolumntype{C}[1]{>{\Centering}m{#1}}
%\renewcommand\tabularxcolumn[1]{C{#1}}
\usepackage[left=2cm,right=2cm,top=2cm,bottom=2cm]{geometry}
\usepackage{subcaption} 
\usepackage{caption}
\usepackage[colorlinks]{hyperref}
\usepackage[svgnames]{xcolor}
\hypersetup{citecolor=DeepPink4}
\hypersetup{linkcolor=DarkRed}
\hypersetup{urlcolor=DarkBlue}
\usepackage{cleveref}

\begin{document}
\setlength{\parindent}{0cm} 


\frenchspacing

\title {Quiz 1 Mona Mohammed} 
\author {CENG 340--Introduction to Environmental Engineering\\
Instructor: Deborah Sills}
%\date {, 2013}
\maketitle

\section *{No More Questions About Vinyl Chloride!}
Trichlorethylene (TCE)--- (C$_2$HCl$_3$)--- is used as a degreasing solvent in many industries and is classified as a suspected carcinogen by the U.S. Environmental Protection Agency (EPA).  The EPA has set an enforceable regulation for TCE, called a maximum contaminant level (MCL), at 0.005 mg/L. 
\begin{enumerate}
\item Prove that 0.005 mg/L equals 5 ppb$_m$. (Hint: assume that the density of water is 1 g/mL). [6 points] 

\vspace{2.3in}
\item What is the MCL of TCE in units of mole/L? [2 points] 

\vspace{1in}
\item If a local tannery dumped 30 g of TCE into a full 300 m$^3$ storage tank, would the resulting concentration of TCE violate the MCL (show your work)? [2 points] 


\end{enumerate}
\end{document}