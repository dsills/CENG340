\documentclass[12pt,letterpaper]{article}
%\documentstyle[11pt]{article}
\usepackage[utf8]{inputenc}
\usepackage{amsmath}
\usepackage{xfrac}
\usepackage{amsfonts}
\usepackage{amssymb}
\usepackage[version = 3]{mhchem}
\usepackage{chemstyle}
%%For Table perhaps%%
%\usepackage{graphics}
\usepackage{graphicx}
\usepackage{epstopdf}
%\usepackage{tabularx,ragged2e,booktabs,caption}
%\newcolumntype{C}[1]{>{\Centering}m{#1}}
%\renewcommand\tabularxcolumn[1]{C{#1}}
\usepackage[left=2cm,right=2cm,top=2cm,bottom=2cm]{geometry}
\usepackage{subcaption} 
\usepackage{caption}
\usepackage[colorlinks]{hyperref}
\usepackage[svgnames]{xcolor}
\hypersetup{citecolor=DeepPink4}
\hypersetup{linkcolor=DarkRed}
\hypersetup{urlcolor=DarkBlue}
\usepackage{cleveref}

\begin{document}
\setlength{\parindent}{0cm} 


\frenchspacing

\title {Prep for Quiz 1 or Extra Credit HW---Environmental Measurements} 
\author {CENG 340--Introduction to Environmental Engineering\\
Instructor: Deborah Sills}
%\date {, 2013}
\maketitle

\section *{Due Date}
Hand in by 5pm Wednesday (if I'm not in my office---215 Dana---, slip it under the door).

\section *{Rationale}
You can turn this in as an extra credit homework assignment that will be included in the calculation of your homework grade.\\

\section *{Learning Goals}
\begin{enumerate}
\item Calculate chemical concentration in units of mass/mass, mass/volume, mole/volume, ppm$_v$, ppm$_m$.
\item Calculate chemical concentration in common constituent units such as hardness, nitrogen, and CO$_2$ equivalents. 
\item Become comfortable using standard environmental measurements.
\end{enumerate}

\section *{Questions}

\begin{enumerate}
\item Vinyl chloride is used to produce polyvinyl chloride (PVC), which is a plastic material used in construction.  Vinyl chloride is classified as a known carcinogen by the U.S. Environmental Protection Agency (EPA), and according to \href{http://water.epa.gov/drink/contaminants/basicinformation/vinyl-chloride.cfm}{their website}, "EPA has set an enforceable regulation for vinyl chloride, called a maximum contaminant level (MCL), at 0.002 mg/L or 2 ppb."  Prove that 0.002 mg/L equals 2 ppb. 

\item (modified from Mihelcic and Zimmerman) The \href{http://water.epa.gov/drink/contaminants/basicinformation/vinyl-chloride.cfm}{EPA regulates nitrate in drinking water with a MCL of 10 mg/L-N}.  Babies under the age of 6 months who drink water with nitrate concentrations higher than the MCL could be come very ill and evn die from a syndrome called "blue-baby syndrome."  If a water sample contains 10 mg NO$_3^{-2}$/L, does it violate the MCL?  Also convert the MCL to units of (a)ppm$_m$, (b) moles/L, and (c) ppb$_m$. 

\item \textbf{Who loves hockey?} (modified from Mihelcic and Zimmerman)\\
Ice resurfacing machines (aka Zambonis) use internal combustion vehicles that give off exhaust containing carbon monoxide (CO) and nitrogen oxides (NO$_x$).  The outdoor air quality 1-h standard of CO is set at 35 mg/m$^3$. Average CO concentrations measured at Lynah Rink (at Cornell University) have been reported to be as high as 115 ppm$_v$ and as low as 35 ppm$_v$.  (1) Should Prof. Sills be concerned about spending 1 h at Lynah Rink, watching Cornell Women's Ice Hockey Team play (and hopefully beat) Northeastern on October 19th?  Assume the temperature equals 20$^0$C. (2) Calculate the partial pressure (in atm) of CO in the rink.  Assume that the atmospheric pressure is 1 atm.

\item What is the molar concentration of 10 grams/liter for each of the following chemicals?
\begin{itemize}
\item NaOH     
\item Na$_2$SO$_4$   
\item K$_2$Cr$_2$O$_7$   
\item KCl  
\end{itemize}

\item 	Express 50 mg/L of  HCO$_3^-$ as:
\begin{itemize}
\item equivalents/liter
\item moles/liter
\item 	milligram/liter as CaCO3
\end{itemize}

\item A lake water sample has the following cation composition:
\begin{itemize}
\item Ca$^{2+}$ = 42 mg/L
\item Na$^{+}$ = 35 mg/L
\item Mg$^{2+}$ = 12 mg/L
\item Mn$^{2+}$ = 10 mg/L
\end{itemize}
Determine the hardness of the water (express the hardness as mg/L of CaCO3).
 
\item (modified from Mihelcic and Zimmerman) Coliform bacteria (for example \emph{E. coli}) are excreted in large numbers in human and animal feces.  Water that meets a standard of less than one coliform per 100 mL is considered safe for human consumption.  Is a 1 m$^3$ water sample that contains 9000 coliforms safe for human consumption?  \textbf{Show your work.}


\item (from Mihelcic and Zimmerman) In 2004, U.S. landfills emitted approximately 6,709 Gg of methane, and wastewater treatment plants emitted 1,758 Gg of methane.  How may Tg  of CO$_2$ equivalents did landfills and wastewater plants emit in 2004.  What percent of the total methane emissions (and greenhouse gas emissions [GHGs]) do these two sources contribute? Total methane emissions in 2004 were 556.7 Tg CO$_2$ equivalents and total GHG emissions were 7074 CO$_2$ equivalents. 


\item Two hundred milliters of a lake water sample is completely evaporated at 104$^0$C in an evaporating dish.  The tare weight of the dish was 12.819 grams.  After evaporation of the sample the weight of the dish plus residue was 13.020 grams.  The dish plus residue was then heated (ignited) at 550$^0$C.  The dish plus residue weighed 12.982 grams after cooling.  Determine (1) total solids content, (2) total inorganic (fixed) solids, and (3)total organic (volatile) solids content of the sample. 


\end{enumerate}

\end{document}