\documentclass[12pt,letterpaper]{article}
%\documentstyle[11pt]{article}
\usepackage[utf8]{inputenc}
\usepackage{amsmath}
\usepackage{xfrac}
\usepackage{amsfonts}
\usepackage{amssymb}
\usepackage[version = 3]{mhchem}
\usepackage{chemstyle}
%%For Table perhaps%%
%\usepackage{graphics}
\usepackage{graphicx}
\usepackage{epstopdf}
%\usepackage{tabularx,ragged2e,booktabs,caption}
%\newcolumntype{C}[1]{>{\Centering}m{#1}}
%\renewcommand\tabularxcolumn[1]{C{#1}}
\usepackage[left=2cm,right=2cm,top=2cm,bottom=2cm]{geometry}
\usepackage{subcaption} 
\usepackage{caption}
\usepackage[colorlinks]{hyperref}
\usepackage[svgnames]{xcolor}
\hypersetup{citecolor=DeepPink4}
\hypersetup{linkcolor=DarkRed}
\hypersetup{urlcolor=DarkBlue}
\usepackage{cleveref}

\begin{document}
\setlength{\parindent}{0cm} 


\frenchspacing

\title {Text Book Correction--Example 2.14, p. 43} 
\author {CENG 340--Introduction to Environmental Engineering\\
Instructor: Deborah Sills}
%\date {, 2013}
\maketitle

\section *{Determination of a Water's Hardness}
Water has the following chemical composition: [Ca$^{2+}$] = 15 mg/L; [Mg$^{2+}$] = 10 mg/L,; [SO$_4^{2-}$] = 30 mg/L.  What is the total hardness in units of mg/L as CaCO$_3$.

\section *{Solution}
Find the contribution of hardness from each \emph{divalent cation}.  Anions and all nondivalent cations to not contribute to hardness.\\

\begin{equation}
\frac{15\, mg\, Ca^{2+}}{{L}}\times\Bigg(\frac{\frac{50 \, g \, CaCO_3}{eq}}{\frac{40 \, g \, Ca^{2+}}{2 eq}}\Bigg) = \frac{38 \, mg}{L}\, as \, CaCO_3
\end{equation}\\

\begin{equation}
\frac{10\, mg\, Mg^{2+}}{{L}}\times\Bigg(\frac{\frac{50 \, g \, CaCO_3}{eq}}{\frac{24 \, g \, Mg^{2+}}{2 eq}}\Bigg) = \frac{42 \, mg}{L}\, as \, CaCO_3
\end{equation}\\

Therefore, the total hardness is 38 + 42 = 80 mg/L as CaCO$_3$.  This water is moderately hard.  \\

Note that if reduced iron (Fe$^{2+}$) or manganese (Mn$^{2+}$) were present they would be included in the hardness calculation.


\end{document}