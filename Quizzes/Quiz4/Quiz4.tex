\documentclass[12pt,letterpaper]{article}
%\documentstyle[11pt]{article}
\usepackage[utf8]{inputenc}
\usepackage{amsmath}
\usepackage{xfrac}
\usepackage{amsfonts}
\usepackage{amssymb}
\usepackage[version = 3]{mhchem}
\usepackage{chemstyle}
%%For Table perhaps%%
%\usepackage{graphics}
\usepackage{graphicx}
\usepackage{epstopdf}
\usepackage{tabularx,ragged2e,booktabs,caption}
%\newcolumntype{C}[1]{>{\Centering}m{#1}}
%\renewcommand\tabularxcolumn[1]{C{#1}}
\usepackage[left=1.5cm,right=1.5cm,top=0.5cm,bottom=2cm]{geometry}
\usepackage{subcaption} 
\usepackage{caption}
\usepackage[colorlinks]{hyperref}
\usepackage[svgnames]{xcolor}
\hypersetup{citecolor=DeepPink4}
\hypersetup{linkcolor=DarkRed}
\hypersetup{urlcolor=DarkBlue}
\usepackage{cleveref}

\begin{document}
\setlength{\parindent}{0cm} 


\frenchspacing


\title {\Large{\textbf{Quiz 4---Water Quality}}\\ \large{CENG 340--Introduction to Environmental Engineering\\
Instructor: Deborah Sills\\ \textbf{25 October, 2013}}}
\author {}
\date {}
\maketitle

\vspace{-0.4 in}
\textbf{\large{Name:}}\\

\begin{enumerate}

\item  Given the following analysis of a raw(untreated) water:\\



\begin{minipage}{\linewidth}
\centering
%\captionof{table}{\textbf{Typical values used in design of water treatment systems} (adapted from our textbook).} \label{tab:title}

%\begin{tabular}{|p{20mm}|p{40mm}|p{50mm}|p{35mm}|}\toprule[1.25pt]
\begin{tabular}{|c|c|c|c|}\toprule[1.25pt]
 Chemical	&  mg/L as chemical 	&  $\mathrm{\frac{EW^*\, as\, CaCO_3}{EW^*\, as\, ion}}$ &  mg/L as CaCO$_3$	\\\midrule
CO$_2$ & 8.8 & 2.27 & 20\\ \hline\
Ca$^{2+}$ & 103 & 2.5 & 258\\ \hline\
Mg$^{2+}$ & 5.5 & 4.12 & 23\\ \hline\
Na$^+$ & 16 & 2.18 & 35\\ \hline\
HCO$_3^-$ & 255 & 0.82 & 209\\ \hline\
SO$_4^{2-}$ & 49 & 1.04 & 51\\ \hline\
Cl$^-$ & 37 & 1.41 & 52\\

\bottomrule[1.25pt]

\end {tabular}\par
\end{minipage}\\

$^*$EW stands for equivalent weight

\begin{enumerate}

\item \emph{(2 points)} Report the the total hardness, carbonate hardness, and non-carbonate hardness in units of mg/L as CaCO$_3$.\\
\vspace{1.6in}


\item \emph{(2 points)} Determine how much lime (in units of mg/L as CaCO$_3$) must be added to remove calcium.

\vspace{0.8in}


\item \emph{(1 points)} After softening with the amount of lime calculated in (b), what is the remaining hardness of the water.

\end{enumerate}



\vspace{3in}



\item \emph{(2 point)} \textbf{Short Answer:} Name one water contaminant (or class of water contaminants) that poses a threat to human health, and state its associated health concern.
\vspace{1.4in}


\item \emph{(3 points)} \textbf{Multiple Choice:} One or more answers may be correct in the following question:\\


Turbidity
\begin{enumerate}
\item is higher in groundwater than in lakes.
\item is used as an indicator of synthetic organic materials.
\item is used as an indicator of the presence of pathogenic microorganisms.
\item is used as an indicator of heavy metals.
\item can be removed by lime precipitation followed by sedimentation.
\item is regulated by the Safe Drinking Water Act.
\end{enumerate} 











\end{enumerate}




\end{document}