\documentclass[12pt,letterpaper]{article}
%\documentstyle[11pt]{article}
\usepackage[utf8]{inputenc}
\usepackage{amsmath}
\usepackage{xfrac}
\usepackage{amsfonts}
\usepackage{amssymb}
\usepackage[version = 3]{mhchem}
\usepackage{chemstyle}
%%For Table perhaps%%
%\usepackage{graphics}
\usepackage{graphicx}
\usepackage{epstopdf}
%\usepackage{tabularx,ragged2e,booktabs,caption}
%\newcolumntype{C}[1]{>{\Centering}m{#1}}
%\renewcommand\tabularxcolumn[1]{C{#1}}
\usepackage[left=2cm,right=2cm,top=0.5cm,bottom=2cm]{geometry}
\usepackage{subcaption} 
\usepackage{caption}
\usepackage[colorlinks]{hyperref}
\usepackage[svgnames]{xcolor}
\hypersetup{citecolor=DeepPink4}
\hypersetup{linkcolor=DarkRed}
\hypersetup{urlcolor=DarkBlue}
\usepackage{cleveref}

\begin{document}
\setlength{\parindent}{0cm} 


\frenchspacing


\title {\Large Quiz 2---Environmental Chemistry} 
\author {CENG 340--Introduction to Environmental Engineering\\
Instructor: Deborah Sills}
\date {September 18, 2013}
\maketitle


\vspace{-0.2 in}
\textbf{Name:}\\


Since the industrial revolution, atmospheric CO$_2$ concentrations have risen from approximately 320 ppm$_v$ to 390 ppm$_v$.  Because the concentration of CO$_2$ in the air is in \emph{equilibrium} with the concentration of dissolved CO$_2$, this increase has led to a change in the aqueous (= dissolved) CO$_2$ concentration in the ocean.\\

Calculate this change in the aqueous concentration of CO$_2$ in the ocean (ignoring effects of salinity).  Assume T = 25 $^0$C; P$_{\mathrm{air}}$ = 1 atm; Henry's constant for CO$_2$, K$\mathrm{_H}$ = 0.0246 $\mathrm{\frac{moles}{L\times atm}}$.\\

Additional information that may be useful: the ideal gas constant R = $8.205\times 10^{-5}\, \mathrm{\frac{m^3 \times atm}{mole \times K}}$, temperature in Kelvin (K) = temperature in Celsius ($^0$C) + 273.15; (2) MW$\mathrm{_C}$ = 12 g/mole and MW$\mathrm{_O}$ = 16 g/mole.





\end{document}