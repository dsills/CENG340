\documentclass[12pt,letterpaper]{article}
%\documentstyle[11pt]{article}
\usepackage[utf8]{inputenc}
\usepackage{amsmath}
\usepackage{xfrac}
\usepackage{amsfonts}
\usepackage{amssymb}
\usepackage[version = 3]{mhchem}
\usepackage{chemstyle}
%%For Table perhaps%%
%\usepackage{graphics}
\usepackage{graphicx}
\usepackage{epstopdf}
%\usepackage{tabularx,ragged2e,booktabs,caption}
%\newcolumntype{C}[1]{>{\Centering}m{#1}}
%\renewcommand\tabularxcolumn[1]{C{#1}}
\usepackage[left=2cm,right=2cm,top=2cm,bottom=2cm]{geometry}
\usepackage{subcaption} 
\usepackage{caption}
\usepackage[colorlinks]{hyperref}
\usepackage[svgnames]{xcolor}
\hypersetup{citecolor=DeepPink4}
\hypersetup{linkcolor=DarkRed}
\hypersetup{urlcolor=DarkBlue}
\usepackage{cleveref}

\begin{document}
\setlength{\parindent}{0cm} 


\frenchspacing

\title {Chemistry Guide} 
\author {CENG 340--Introduction to Environmental Engineering\\
Instructor: Deborah Sills\\Fall 2013}
 
\maketitle

\section {Environmental Measurements and Units}
To calculate the transport and transformation of chemicals in the environment, you need to be able to use the units presented here.  This material does not lend itself to presenting underlying fundamental concepts.  However, without mastering the ability to manipulate units correctly you won't be able to apply the fundamentals that are covered later in the course.\\

Make sure you can solve all of the problems on the Quiz 1 Prep document; Problem 3 in Pset1; Quiz 1; the in-class problem from 9 September, titled ``What are PVC pipes made from''; and Problem 1 in PSet 2.  

\subsection {Concentration Units}
The concentration of a chemical determines degradation rates (except for zero order reactions) and the transport of many reactions (excluding zero order).\\  

Concentration Units you need to know for this class:

\begin{enumerate}
\item In Water:
\begin{enumerate}
\item You need to be able to convert from a mass concentration (e.g., mg/L, $\mathrm{\mu g/m^3}$) to a mole concentration (e.g., mole/L, mmole/mL).

\item ppm$\mathrm{_m}$ = g of \emph{i} in $10^6$ g total\\

ppm$\mathrm{_m}$ = $\mathrm{\frac{m_i}{m_{total}}\times 10^6}$\\

ppm$\mathrm{_m}$ = $\mathrm{\frac{mg}{L}}$, but only only only \textbf{in water}.\\

\item ppb$\mathrm{_m}$ = g of \emph{i} in $10^9$ g total\\

ppb$\mathrm{_m}$ = $\mathrm{\frac{m_i}{m_{total}}\times 10^9}$\\

ppb$\mathrm{_m}$ = $\mathrm{\frac{\mu g}{L}}$, but only only only \textbf{in water}.\\

ppb$\mathrm{_m}$ = ppm$\mathrm{_m}\times 10^3$\\

Note that these units should only be used for aqueous or solid (e.g., soil) phases.  \emph{In this class, we only use ppm$\mathrm{_m}$ and ppb$\mathrm{_m}$ for aqueous concentrations---i.e., concentrations in water}.

\item Normality: Used for concentration of acid:\\

1 N = 1 $\mathrm{\frac{eq}{L}}$, and\\

1 $\mathrm{\frac{eq}{L}}$ of acid = 1 $\mathrm{\frac{mole}{L}}$ of [H$^+$].\\

Review example 2.11 on p. 39 of the textbook.

\end{enumerate}

\item In Gas\\
\begin{enumerate}

\item PV = nRT\\

where P is pressure, V is volume, n is number of moles, R is the ideal gas constant, and T is temperature in Kelvins.\\

\item ppm$\mathrm{_v}$ = $\mathrm{\frac{V_i}{V_{total}}\times 10^6}$\\

where $\mathrm{\frac{V_i}{V_{total}}}$ is the volume fraction.\\

In addition,\\

ppm$\mathrm{_v}$ = $\mathrm{\frac{P_i}{P_{total}}\times 10^6 = \frac{n_i}{n_{total}}\times 10^6}$\\

where P$_i$ is the partial pressure of the chemical species, P$_{\mathrm{total}}$ is the total pressure of the gas, n$_i$ is the number of moles of the chemical species, and n$_{\mathrm{total}}$ is the total number of moles.\\

Note that partial pressure, P$_i$, is a concentration of the chemical species in the gas phase.

\item Convert from ppm$\mathrm{_v}$ to mass per volume:\\

$\mathrm{\frac {\mu g}{L}= ppm_v\times MW \times \frac{1000\, P}{RT}}$\\

where MW is the molecular weight of the chemical species, R = $\mathrm{0.08205\, \frac{L\times atm}{mole\times K}}$, T is the temperature degrees K, and 1000 is a conversion factor (1000 L = 1 m$^3$).\\

\item Global Warming Potential (GWP):\\

Multiplier used to compare emissions of greenhouse gases (GHGs) to carbon dioxide over 100 years. 

GWP$_{\mathrm{CH_4} = 25\times GWP_{\mathrm{CH_4}}}$, where GWP is in units of \textbf{mass} of CO$_2$ equivalents.\\



\end{enumerate}

\item In Solids (e.g., soil):\\

In this class we've only used mass per mass units for sorption isotherms---e.g., $\mathrm{\frac{mg\, chemical\, species}{g\, soil}}$.

\end{enumerate}
\section {Environmental Chemistry}
In this section, we covered three main topics:

\begin{enumerate}
\item Mass balance using stoichiometry
\item Chemical Equilibrium
\item Rates of Reactions\\ 
\end{enumerate}

Details on what you're expected to know on these three topics are included below.  

\subsection {Mass Balances of Chemical Reactions Using Stoichiometry}
The fundamental principle here is that the number of each element must be the same on either side of a chemical equation. For example in the following equation that describes the combustion of methane (CH$_4$)  , each mole of CH$_4$ combusted required two moles of O$_2$. 

\begin{align}
\cee{CH_4 + 2O_2  &-> CO_{2} + 2H_2O}
\end{align}

Refer to your notes from Monday, 9 September, where we calculated that mass O$_2$ required for combustion of 500 kg of CH$_4$ gas, as well as the mass of CO$_2$ produced.

In addition, refer to Problem 6 in PSet 2.  This type of problem will appear on the midterm, or on the final---you can count on it.

\subsection {Chemical Equilibrium}

Chemical equilibrium represents the end point of a reaction, and is useful in environmental engineering when we're analyzing fast reactions.\\

Important fast reactions include chemical partitioning between water and air, acid--base reactions, oxidation-reduction reactions (will not be part of Midterm 1), precipitation--dissolution reactions, and chemical partitioning between water and soil and air and soil (we only covered water and soil).\\

A short description, including why the reaction type is important, of each type of equilibrium relationship is included below, including references to in-class and homework problems you should make sure you can do before the exam.\\

For instructions on how to approach Equilibrium Problems, refer to the handout I gave you on Friday, 20 September (handout is posted as part of lecture notes for 9/20 on the course schedule). 


\begin{enumerate}
\item Air--water partitioning or Henry's Law:\\

Henry's law is empirical, meaning that it is based on experimental observations.  There are fundamental reasons why partitioning occurs, but you would need to take a mini-course in thermodynamics before we could cover these fundamentals.  Anyone interested, please contact me to discuss further.\\

Air--water partitioning is important, because the equilibrium relationship that governs this phenomena allows us to calculate the concentration of a chemical in the water phase if we know the concentration of the chemical in the gas phase, and vice versa.  Why does this matter?\\

I'll use CO$_2$ as an example.  Scientists have known for a while that increased atmospheric concentrations of CO$_2$ are leading to global warming.  But Henry's law dictates that if CO$_{2(g)}$ increases so does CO$_{2(aq)}$.  And since higher CO$_{2(aq)}$ leads to lower pH and acidification of surface waters, aquatic systems may be impacted.\\

I'm not sure what else to say to help you know why you are doing these type of calculations, but if a person showers with water that is contaminated with a volatile compound (such as vinyl chloride or dry cleaning fluid), he will be exposed to it in the shower, even if he doesn't drink the water.  And Henry's Law governs this type of exposure to toxic volatile chemicals.\\

\textbf{Refer to Problem 4 and Problem 5 in Pset2 and Quiz 2} for practice.\\

In addition always make sure you check that you write the equilibrium relationship in such a way that the units match up with the units of Henry's Constant, K$\mathrm{_H}$.  \textbf{Units of Henry's constant, may be given in the following units:} $\mathrm{\frac{mole}{L\times atm}}$, $\mathrm{\frac{L\times atm}{mole}}$, $\mathrm{\frac{\frac{mole}{L}\, gas}{\frac{mole}{L}\, aq}}$.

\item Acid--Base Reactions:\\

Acid base dissociation reactions are also based on fundamental thermodynamic principles.  Again if you're interested in learning about the governing principles, contact me, or take Water Chemistry, which is taught by Prof. Higgins, in your senior year.\\

\begin{enumerate}

\item \textbf{Strong Acids}\\
Strong acids (e.g., $\mathrm{H_2SO_4}$ and HCl) dissociate completely, as shown in the following equation:\\

\begin{align*}
\cee{HCl}\, \rightarrow \, H^+ \, + \,  Cl^-
\end{align*}

In the case of HCl, this means that every mole of HCl added to water produces one mole of H$^+$.  In addition, the ``undissociated species'' is HCl and the ``dissociated species'' is Cl$^-$.\\

Note: I will tell you if an acid is strong or weak.\\

\item \textbf{Weak Acids:}

Weak acids do not dissociate completely and the ratio of the undissociated species and dissociated species is governed by pH and the equilibrium constant, aka acid dissociation constant---or K$\mathrm{_a}$.  For example, ammonium dissociation is described by the following equilibrium relationship:

\begin{align*}
\cee{[NH_4^+] &<=>[K_a] [NH_3] + [H^+]}
\end{align*}

In addition,

\begin{align*}
\mathrm{K_a = \frac{[NH_3][H^+]}{[NH_4^+]}}
\end{align*}

Note that for this example, the term ``Total Ammonia'' refers to $\mathrm{[NH_3] + [NH_4^+]}$, the term  ``undissocated species'' refers to $\mathrm{[NH_4^+]}$, and the term ``dissociated species'' refers to $\mathrm{[NH_3]}$.\\

To solve weak acid problems refer to the in-class problem titled ``Fish Kill'', which was handed out on 11 September and Problem 4 in Pset 3 (assume pH = 7).\\

\item \textbf{Alkalinity}\\

Alkalinity is defined as the ability of a solution to neutralize acid.  The carbonates---bicarbonate, [HCO$_3^-$]; and carbonate, [CO$_3^{2-}$] are the predominant source of alkalinity in natural waters.  The formal definition of alkalinity is as follows:\\

Alkalinity (eq/L) = [HCO$_3^-$] + 2[CO$_3^{2-}$] + [OH$^-$] - [H$^+$]\\

Alkalinity is important because it represents the ability of a natural system to neutralize acid rain, for example; and the ability of the slurry in a digester to neutralize acid produced during anaerobic digestion.  If the pH of the slurry in a digester falls below pH=7, the digester fails!\\

Units of alkalinity are often reported as mg/L of CaCO$_3$.  Since the molecular weight of CaCO$_3$ is 100 g/mole and CO$_3^{2-}$ has 2 equivalents per mole, you can convert from equivalents to mg/L CaCO$_3$.

To practice alkalinity problems refer to Problem 3 in PSet 3.  Note that in the FE, ``approximate alkalinity'' means: ignore [OH$^-$] and [H$^+$].

\end{enumerate}

\item Precipitation--Dissolution Reactions\\

Precipitation dissolution equilibria govern how much of two aqueous species will precipitate into a solid.  Since precipitation is an equilibrium relationship and equilibrium refers to the \textbf{end point of a reaction}, you should always use the final concentrations on the aqueous species in these calculations.\\

Precipitation reactions are important for a couple of reasons.  First, they can be used to remove heavy metals from acid mine drainage (or other solutions polluted with metals), which often only require pH reduction which provides the [OH$^-$] needed, as shown in the following reaction:\\

\begin{align*}
\cee{[Cd(OH)_2]  &<=>[K_{sp}] [Cd^{2+}] + 2[OH^-]}
\end{align*}

Precipitation--dissolution equilibrium relationships can also be used to evaluate the need for water softeners and aid in the design of water softeners.  We'll cover this topic again when we cover water treatment, after Fall Break.\\

To practice precipitation--dissolution problems refer to two in-class problems handed out on 16 September and 18 September, as well as Problems 1 and 2 in PSet 3.

\item Sorption or Soil--Water Partitioning\\

Similar to air-water partitioning, equilibrium constants for soil-water partitioning are based on experimental observations. The governing equilibrium relationship is as follows:

\begin{align*}
\cee{q  &<=>[K]C^{\sfrac{1}{n}}}
\end{align*}

where q = mass of chemical sorbed per mass of soil or solid at equilibrium ($\mathrm{\sfrac{mg}{g}}$),\\

C = concentration of chemical in the aqueous phase at equilibrium ($\mathrm{\sfrac{mg}{L}}$),\\

K = Freundlich isotherm soil-water partition coefficient(($\mathrm{\sfrac{mg}{g}}$)($\mathrm{\sfrac{L}{mg})}$), and\\

$\mathrm{\sfrac{1}{n}}$ = Freundlich isotherm intensity parameter (unitless).\\


For more on sorption refer to Lab 3 and Problem 5 in PSet3.

\end{enumerate}

\subsection {Kinetics:}

Kinetics govern how fast reactions occur, and are useful for relatively slow reactions.  Since this topic was covered again under batch reactors, I will not cover it here.

For practice, refer to Lab2, Lab3--Part1, Problem 8 in PSet2, Problem 6 in PSet3, and Problems 3, 6, 7, and 8 in PSet4.










\end{document}