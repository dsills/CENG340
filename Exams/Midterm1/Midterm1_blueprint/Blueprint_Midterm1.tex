\documentclass[12pt,letterpaper]{article}
%\documentstyle[11pt]{article}
\usepackage[utf8]{inputenc}
\usepackage{amsmath}
\usepackage{xfrac}
\usepackage{amsfonts}
\usepackage{amssymb}
\usepackage[version = 3]{mhchem}
\usepackage{chemstyle}
%%For Table perhaps%%
%\usepackage{graphics}
\usepackage{graphicx}
\usepackage{epstopdf}
%\usepackage{tabularx,ragged2e,booktabs,caption}
%\newcolumntype{C}[1]{>{\Centering}m{#1}}
%\renewcommand\tabularxcolumn[1]{C{#1}}
\usepackage[left=2cm,right=2cm,top=2cm,bottom=2cm]{geometry}
\usepackage{subcaption} 
\usepackage{caption}
\usepackage[colorlinks]{hyperref}
\usepackage[svgnames]{xcolor}
\hypersetup{citecolor=DeepPink4}
\hypersetup{linkcolor=DarkRed}
\hypersetup{urlcolor=DarkBlue}
\usepackage{cleveref}

\begin{document}
\setlength{\parindent}{0cm} 


\frenchspacing

\title {Midterm 1: Exam Blueprint} 
\author {CENG 340--Introduction to Environmental Engineering\\
Instructor: Deborah Sills\\}
 
\maketitle


\section *{Exam Format}
Exam will include mostly problems, but will also include some short answer and multiple choice questions.  When solving problems, make sure you write down what you know, and, if you're stuck, describe what you are thinking and how you are approaching the problem.\\

You should bring a single-side page with equations and unit conversions (do not include words) to use in the exam.  (``Alk'' is the closest thing to a word that should be on the exam.) You will be asked to submit your equation sheet with your exam. I included a list of useful equations and unit conversions at the end of this document.

\section *{Learning Goals}
\subsection *{State of the Planet--Chapter 1 in the text book}

\begin{enumerate}
\item Describe three environmental challenges presented in the chapter.
\item Describe how one global environmental challenges may affect the civil and environmental engineering professions.


\end{enumerate}

\subsection *{Environmental Measurements}
\emph{Relevant sections in Chapter 2 of the textbook: 2.1--2.4, 2.5.1--2.5.3}\\

\textbf{Topics Covered:}\\
\begin{enumerate}
\item Calculate chemical concentrations in units of mass/mass, mass/volume, mole/volume, mole/mole, volume/volume, ppm$_v$, ppm$_m$, and partial pressure.
\item \textbf{Demonstrate that you know when to use units of ppm$_m$ (for liquid concentrations), ppm$_v$ (for gas concentrations).}
\item Calculate chemical concentration in common constituent units such nitrogen species (e.g. NO$_3^{2-}$ in units of ``as N'', greenhouse gases (in units of CO$_2$ equivalents), and alkalinity in units of eq/L and ``as CaCO$_3$.'' 
\item Describe and, given the appropriate data, calculate concentration of the following types of solid particles in water : TS, TSS, TDS, VS, FS, FSS, VSS, FDS, VDS.
\end{enumerate}

\subsection *{Water Quality}
\emph{Relevant sections of Chapter 10 in the textbook: 10.1-10.3}\\

\textbf{Topics Covered:}\\

\begin{enumerate}
\item Describe the characteristics of water: physical (Table 10.2), chemical (organic and inorganic), and biological (viruses, bacteria, protozoa). Refer to PPT file handed out on Monday, 9/9 (also posted on the schedule page of the course website.
\item Be able to characterize a contaminant as physical, chemical, or biological.
\item Use Table 10.8 from the text book (would be provided in an exam) to compare a given concentration of a pollutant to the regulated concentration.
\end{enumerate}

\subsection *{Water Chemistry}

\emph{Relevant sections of Chapter 3 in the textbook:3.1 (approaches in environmental chemistry), 3.3 (stoichiometry), 3.6 (air-water partitioning), 3.7 (acid--base and alkalinity), 3.9 (precipitation--dissolution), pp. 76--79 (sorption), 3.11(kinetics)}\\

\textbf{Topics Covered:}\\

\begin{enumerate}
\item Apply the law of conservation of mass to write balanced chemical equations and to calculate masses of reactants and products.
\item Identify which chemical approach--- equilibrium or kinetic--- should be applied to a given environmental problem.
\item Apply equilibrium relationships to calculate chemical concentrations of pollutants in air, water, and soil using the following four types of reactions: 
\begin{enumerate}
\item Water--gas partitioning, based on Henry's Law.  Be able to use Henry's constant, K$\mathrm{_H}$, in the following units: $\mathrm{\frac{mole}{L\times atm}}$, $\mathrm{\frac{L\times atm}{mole}}$, $\mathrm{\frac{\frac{mole}{L}\, gas}{\frac{mole}{L}\, aq}}$
\item Acid-Base
\item Precipitation Dissolution
\item Solid-Water Partitioning (sorption)
\end{enumerate}
\item State, in differential form, a change in concentration with time according to zero-, first-, and second-order reaction kinetics.
\item Solve (or integrate) the differential form of zero-, first-, and second-order reaction kinetics.
\item Given sufficient data, calculate C, C$_0$, k, or t for zero-, first-, or second-order reactions. 

\end{enumerate}

\subsection *{Mass Balance with (or without) reactions}

\emph{Relevant sections of Chapter 4 in the textbook:pp. 106--127}\\

\textbf{Topics Covered:}\\
\begin{enumerate}
\item Describe the term ``mass balance'' in words and write the general mass balance equation.
\item Prepare a labeled diagram of reactor systems that includes inputs, outputs, control volume, known and unknown variables.
\item Derive equations for batch and CMFR reactors that describe C as f(time), or t as f(C, C$_0$).
\item Calculate C, t, HRT, V, Q, or k for CMFR, batch, and PFR reactors, given sufficient data.
\item Determine from a problem description, reactor type (batch, PFR, CSTR), and steady-state or non-steady-state conditions. And modify the general mass balance equation accordingly.
\item Compare and explain the efficiency of a PFR and CMFR for zero and first order reactions.  
\item Sketch concentration vs. time graphs for a conservative tracer in ideal and real CMFR and PFR reactors. 
\end{enumerate}

\section *{Useful Equations and Unit Conversions:}

\begin{equation*}
\mathrm{[H^+][OH^-] = 10^{-14}}
\end{equation*}

\begin{equation*}
\mathrm{pH = -log[H^+]}
\end{equation*}

\begin{equation*}
\mathrm{Alk (eq/L) = [HCO_3^-] + 2[CO_3^{2-}] + [OH^-] - [H^+]}
\end{equation*}

\begin{equation*}
\mathrm{PV = nRT}
\end{equation*}

\begin{equation*}
\mathrm{K = 273 + ^0C}
\end{equation*}

\begin{equation*}
\mathrm{R = 0.08205\, \frac{L\times atm}{mole\times K}}
\end{equation*}

\begin{equation*}
\mathrm{\frac{P_i}{P_{total}} = \frac{V_i}{V_{total}} = \frac{n_i}{n_{total}} = ppm_v\times 10^6}
\end{equation*}

\begin{equation*}
\mathrm{1000\, L = 1\, m^3}
\end{equation*}

\end{document}