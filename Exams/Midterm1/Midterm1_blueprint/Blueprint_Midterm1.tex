\documentclass[12pt,letterpaper]{article}
%\documentstyle[11pt]{article}
\usepackage[utf8]{inputenc}
\usepackage{amsmath}
\usepackage{xfrac}
\usepackage{amsfonts}
\usepackage{amssymb}
\usepackage[version = 3]{mhchem}
\usepackage{chemstyle}
%%For Table perhaps%%
%\usepackage{graphics}
\usepackage{graphicx}
\usepackage{epstopdf}
%\usepackage{tabularx,ragged2e,booktabs,caption}
%\newcolumntype{C}[1]{>{\Centering}m{#1}}
%\renewcommand\tabularxcolumn[1]{C{#1}}
\usepackage[left=2cm,right=2cm,top=2cm,bottom=2cm]{geometry}
\usepackage{subcaption} 
\usepackage{caption}
\usepackage[colorlinks]{hyperref}
\usepackage[svgnames]{xcolor}
\hypersetup{citecolor=DeepPink4}
\hypersetup{linkcolor=DarkRed}
\hypersetup{urlcolor=DarkBlue}
\usepackage{cleveref}

\begin{document}
\setlength{\parindent}{0cm} 


\frenchspacing

\title {Midterm 1: Exam Blueprint} 
\author {CENG 340--Introduction to Environmental Engineering\\
Instructor: Deborah Sills\\Not a final document}
 
\maketitle



\section *{Learning Goals}
\subsection *{Chapter 1}
\begin{enumerate}
\item Describe three global environmental challenges presented in the chapter.
\item Describe how one global environmental challenges may affect the civil and environmental engineering professions.


\end{enumerate}

\subsection *{Chapter 2 (relevant sections: 2.1--2.4, 2.5.1--2.5.3)}
\begin{enumerate}
\item Calculate chemical concentrations in units of mass/mass, mass/volume, mole/volume, mole/mole, volume/volume, ppm$_v$, ppm$_m$, and partial pressure.
\item Demonstrate that you know when to use units of ppm$_m$ (for liquid concentrations), ppm$_v$ (for gas concentrations).
\item Calculate chemical concentration in common constituent units such as hardness (in units of eq/L and mg/L as CaCO$_3$), nitrogen (in units of "as N") , and greenhouse gases (in units of CO$_2$ equivalents). [We will cover alkalinity in Ch.3.].
\item Describe and, given the appropriate data, calculate concentration of the following types of solid particles in water : TS, TSS, TDS, VS, FS, FSS, VSS, FDS, VDS.
\end{enumerate}

\subsection *{Chapter 10.1-10.3}
\begin{enumerate}
\item Describe the characteristics of water: physical (Table 10.2), chemical (organic and inorganic), and biological (viruses, bacteria, protozoa). Refer to PPT file handed out on Monday, 9/9.
\item Use Table 10.8 from the text book (would be provided in an exam) to compare a given concentration of a pollutant to the regulated concentration.
\end{enumerate}

\subsection *{Chapter 3 (3.1, 3.3, 3.5, 3.6)}

\begin{enumerate}
\item Apply the law of conservation of mass to chemical equations to calculate masses of reactants and products.
\item Identify which chemical approach--- equilibrium or kinetic--- should be applied to a given environmental problem.
\item Apply equilibrium relationships to calculate chemical concentrations of pollutants in air, water, and soil.
\item Apply the following four types of equilibrium relationships: 
\begin{enumerate}
\item Henry's Law for water gas partitioning.  Be able to use K$\mathrm{_H}$ in the following units: $\mathrm{\frac{mole}{L\times atm}}$, $\mathrm{\frac{L\times atm}{mole}}$, $\mathrm{\frac{\frac{mole}{L}\, gas}{\frac{mole}{L}\, aq}}$
\item Acid-Base
\item Precipitation Dissolution
\item Solid-Water Partitioning (sorption)
\end{enumerate}

\end{enumerate}

\end{document}