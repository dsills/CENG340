\documentclass[12pt,letterpaper]{article}
%\documentstyle[11pt]{article}
\usepackage[utf8]{inputenc}
\usepackage{amsmath}
\usepackage{xfrac}
\usepackage{amsfonts}
\usepackage{amssymb}
\usepackage[version = 3]{mhchem}
\usepackage{chemstyle}
%%For Table perhaps%%
%\usepackage{graphics}
\usepackage{graphicx}
\usepackage{epstopdf}
\usepackage{tabularx,ragged2e,booktabs,caption}
%\newcolumntype{C}[1]{>{\Centering}m{#1}}
\renewcommand\tabularxcolumn[1]{C{#1}}
\usepackage[left=2cm,right=2cm,top=2cm,bottom=2cm]{geometry}
\usepackage{subcaption} 
\usepackage{caption}
\usepackage[colorlinks]{hyperref}
\usepackage[svgnames]{xcolor}
\hypersetup{citecolor=DeepPink4}
\hypersetup{linkcolor=DarkRed}
\hypersetup{urlcolor=DarkBlue}
\usepackage{cleveref}
\usepackage{enumerate}

\begin{document}
\setlength{\parindent}{0cm} 


\frenchspacing

\title {Midterm 1} 
\author {CENG 340--Introduction to Environmental Engineering\\
Instructor: Deborah Sills}
\date {September 25, 2013}
\maketitle

As you work through the exam, please write down what you know (in equation form when possible)
\begin{enumerate}
\item A small well-mixed pond has been contaminated with 10 mg/L of a toxic chemical.  The chemical is conservative.  The pond volume is 2500 m$^3$ and has a negligible inflow and outflow (before treatment).  To rid the pond of the contaminant an environmental engineer decided to flush clean water through the pond at a rate of 400 $\mathrm{\frac{m^3}{day}}$.  Determine how long it will take to reduce the chemical concentration to 5 percent of its original value.


\item During a hydrofracking operation the surfactant 2-butoxyethanol was accidentally released into an adjacent river.  The gas company claims that since 2-butoxyethanol is biodegradable the problem is of no concern.  The \emph{unbalanced} reaction for aerobic degradation of 2-butoxyethanol is as follows:

\begin{enumerate}
\item Why might the engineers be wrong?  What environmental concern may occur?

\item If the 2-butoxyethanol is contained in a xx by yy by 4 s deep area, and the dissovled oxygen concentration is 8 mg/L.  how much oxygen must be added to the system?  

\item You've been asked to order a tank to hold the oxygen that will be sued over the course of treatment.  What size (volume) tank do you need to store the oxygen.


\end{enumerate}
\item Short Answer:

\begin{enumerate}

\item What is alkalinity (in words)?

\item Why is alkalinity important?  Name one phenomena or system (natural or engineered) where alkalinity may play role.
\item For what kind of reactions (in general) is it appropriate to use equilibrium?


\end{enumerate}

\item Last week (during parents weekend), a Bucknell parent approached me and asked me how to remove manganese from a river on their property that's a result of acid-mine drainage.  I told them that lime (which is fairly inexpensive) would be added to the river water to precipitate 

\end{enumerate}
\end{document}