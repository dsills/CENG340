\documentclass[a4paper, 12pt]{article}
\usepackage[utf8]{inputenc}
\usepackage{amsmath}
\usepackage{amsfonts}
\usepackage{amssymb}
\usepackage[version = 3]{mhchem}
\usepackage{chemstyle}
%%For Table perhaps%%
%\usepackage{graphics}
\usepackage{graphicx}
\usepackage{epstopdf}
\usepackage{tabularx,ragged2e,booktabs,caption}
%\newcolumntype{C}[1]{>{\Centering}m{#1}}
\renewcommand\tabularxcolumn[1]{C{#1}}
\usepackage[top=.9in, bottom=.9in, left=1in, right=1in]{geometry}

\begin{document}
\title {\textbf{CENG 340 --Introduction to Environmental Engineering}} 

%\author {Deborah Sills}
%\date {9 February 2013}
\maketitle

 {\bf Instructor:} Deborah Sills
 
 {\bf Prerequisites:}
 
 {\bf Office Hours:}
 
 {\bf Text Book}
 

%\item \large \emph {\bf{Differential equations used to calculate CO$_2$ requirements for algal growth and losses to the atmosphere}}\\

\section *{Course Goals}
\begin{enumerate}
\item Integrate and apply relevant content from previous courses\textemdash chemistry, fluid mechanics, calculus, and statistics\textemdash to formulate and solve environmental engineering problems.
\item Identify, formulate, and solve engineering problems.
\item Apply the principles of material and energy balances to analyze environmental problems.
\item Understand the fundamental factors that affect the selection, design, and operation of pollution treatment technologies.
\item Identify appropriate technologies for treatment of specific pollutants present in water, soil, and air.
\item Design basic components of water and wastewater treatment reactors.
\item Analyze engineering designs through the principles of Green Engineering.
\item Develop an appreciation of the impact of civil \& environmental engineering on society and the environment.
\item Improve your ability to present and analyze experimental data. 
\item Improve your ability to communicate technical information in both written and oral form.
\item Develop an appreciation of the value of independent learning.

\end{enumerate}


\section *{Course Materials}
\subsection *{From the Catalog}
An introduction to the fundamentals of environmental engineering and science such as chemistry, microbiology, mass balance, and reactor theory. Application of fundamental concepts to environmental engineering includes water quality, water and wastewater treatment, solid and hazardous waste, air pollution, greenhouse gases and climate change. This course includes hands-on laboratory component with a focus on experiential learning. Prerequisite: ENGR 222 or permission of the instructor.
\subsection *{From Me??}
This course will cover basic fundamental topics as dicussed in the catalog description.  But this course is also an opportunity for you to look critically at some of humanity's big problems.  For example, here in the U.S., we face challenges associated with crumbling, out-of-date civil and environmental engineering infrastructure, and, in certain regions, looming problems of water scarcity.Worldwide, 2.5 billion people lack sanitation and xx billion lack safe drinking water.  And to make things even more interesting, climate change is expected to exacerbate some of the these problems.  I hope that over the semester we can begin to formulate ways engineers might analyze these problems and come up with innovative designs and solutions.  







%\pagebreak




\end{document}  