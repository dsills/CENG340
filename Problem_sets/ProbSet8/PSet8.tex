\documentclass[12pt,letterpaper]{article}
%\documentstyle[11pt]{article}
\usepackage[utf8]{inputenc}
\usepackage{amsmath}
\usepackage{xfrac}
\usepackage{amsfonts}
\usepackage{amssymb}
\usepackage[version = 3]{mhchem}
\usepackage{chemstyle}
%%For Table perhaps%%
%\usepackage{graphics}
\usepackage{graphicx}
\usepackage{epstopdf}
\usepackage{tabularx,ragged2e,booktabs,caption}
\newcolumntype{C}[1]{>{\Centering}m{#1}}
\renewcommand\tabularxcolumn[1]{C{#1}}
\usepackage[left=2cm,right=2cm,top=1cm,bottom=1cm]{geometry}
\usepackage{subcaption} 
\usepackage{caption}
\usepackage[colorlinks]{hyperref}
\usepackage[svgnames]{xcolor}
\hypersetup{citecolor=DeepPink4}
\hypersetup{linkcolor=DarkRed}
\hypersetup{urlcolor=DarkBlue}
\usepackage{cleveref}
\usepackage{enumerate}

\begin{document}
\setlength{\parindent}{0cm} 


\frenchspacing


\title {\Large{\textbf{Problem Set 8}}\\ \large{CENG 340--Introduction to Environmental Engineering\\
Instructor: Deborah Sills\\ \textbf{November 6, 2013}}}

\author {}
\date {}
\maketitle

\vspace{-1in}




\section *{Due Date}
Friday, 15 November, by 5pm.  Bring assignments to my office, or bring them to class on Friday morning.  I'll leave an envelope taped to my door, in case I'm not in my office.

\section *{Learning Goals}
\begin{enumerate}

\item Analyze and evaluate results from BOD tests and theoretical oxygen demand calculations.
\item Apply mass balances and reactor modeling principles to model oxygen demand in the environment.
\item Apply mass balance to evaluate the impact of dewatering on the volume of sludge disposed at a wastewater treatment plant.
\item Describe how food webs, bioaccumulation, and bioconcentration contribute to mercury accumulation in humans. 

\end{enumerate}

\section *{Relevant Sections in the Book}
4.1 (mass balances) and 5.4 (oxygen demand); 5.3.2, 5.3.3, and 5.6 (foodwebs and bioaccumulation)

\section *{Problems}
\begin{enumerate}


\item \emph{(12 pt)} \textbf{BOD} The organic concentration in a water sample, measured as BOD is 4 mg/L.  If the BOD reaction rate is 0.3 day$^{-1}$, what will be the concentration of organic matter remaining at the end of 5 days?  How much oxygen will be used in this period to oxidize the waste? \emph{(Answer: 0.9 mg/L BOD remaining; 3.1 mg/L O$_2$ consumed)}
 
\item \emph{(12 pt)}  \textbf{BOD Approximation} A water sample has a BOD$_5$ of 10 mg/L. The sample was diluted by ten with dilution water and put in a BOD bottle.    Initial dissolved oxygen concentration in the BOD bottle was 8 mg/L.  The sample also contained 2 mg/L NH$_3$-N (before dilution).  Assume the sample was typical domestic sewage (i.e., assume that BOD$_5$ = $\frac{2}{3}$BOD$_U$, or that k = 0.2-0.3 day$^{-1}$).  If the bottle was left sealed for a very long time what was the final DO in the BOD bottle? \emph{(Answer: Values between 5 and 6 mg/L are acceptable.)}


\item \emph{(14 pt)} \textbf{Review your reactor modeling skills} For modeling purposes we need to determine the BOD decay rate for a river.  An experiment was conducted in which two sample were taken from the river at two points separated by a distance of 1 kilometer, and 5-day BOD tests were conducted with both samples in the laboratory.  The sample drawn from Point A (upstream) has a BOD$_5$ = 7.2 mg/L.  The sample drawn from Point B (downstream) has a BOD$_5$ = 3.9 mg/L.  The river exhibits plug flow behavior (i.e., can be modeled as a plug flow reactor), has an average cross-sectional area of 10 m$^2$, and a volumetric flow rate of 100 $\mathrm{\frac{m^3}{h}}$   Determine the “river” BOD decay rate, k$\mathrm{_r}$. \emph{(Answer: 0.147 day$^{-1}$)}


\item \emph{(12 pt)} \textbf{Temperature Effects} A BOD test is performed on an undiluted sample at 20 $^0$C and 35 $^0$C.  BOD$_5$ at 20 $^0$C was 4.15 mg/L.  BOD$_5$ at 35 $^0$C was 6.56 mg/L.  From this data determine the BOD$_U$ of the sample.  Assume that nitrification was inhibited and that the temperature correction factor ($\theta$) equals 1.05.

\item \emph{(12 pt)} \textbf{Review your mass balance skills} The town of Pittsburgh discharges 0.126 $\mathrm{\frac{m^3}{s}}$ of treated wastewater into Cherry Creek.  The BOD$_5$ of the wastewater is 34 $\mathrm{\frac{mg}{L}}$.  Cherry Creek has a 10-year, 7-day low flow of 0.126 $\mathrm{\frac{m^3}{s}}$.  Upstream of the the wastewater outfall from Pittsburgh, the BOD$_5$ is 1.2 mg/L.  The BOD rate constants \emph{k} are 0.222 d$^{-1}$ and 0.090 d$^{-1}$ for the wastewater and the creek, respectively.  The temperature of both the creek and the municipal wastewater is 20 $^0$C. Calculate the initial ultimate BOD (L$_0$) after mixing. \emph{(answer:27 mg/L)}

\item \emph{(14 pt)} \textbf{Review your mass balance skills} A meat processing wastewater containing 2100 $\mathrm{\frac{g}{m^3}}$ BOD$_5$ (at 20 $^0$C test conditions) is to be discharged to a stream.  The minimum stream flow rate (95 $\mathrm{\frac{m^3}{s}}$) occurs in January when the water temperature is 6 $^0$C and maximum temperature occurs in July (26$^0$C) when the flow rate is 175 $\mathrm{\frac{m^3}{s}}$.  If the maximum in-stream BOD$_5$  value is to be 0.5 $\mathrm{\frac{g}{m^3}}$ at ambient temperatures, determine the necessary extent of treatment (i.e. the required \% removal of BOD) for a wastewater flow of 0.2 $\mathrm{\frac{m^3}{s}}$.  Assume the reaction rate constant is 0.2 d$^{-1}$ at 20$^0$C and the temperature coefficient $\theta$ is equal to 1.05.  You may assume the upstream BOD$\mathrm{_u}$ is negligible. \emph{(answer:Necessary treatment required: 82\% BOD removal)}

\item \emph{(12 pt)} \textbf{Review your mass balance skills---sludge dewatering at Milton} The Milton WWTP used to thicken their sludge with gravity settling followed by dissolved air flotation (DAF) only.  Assume that this resulted in 33 $\mathrm{\frac{m^3}{day}}$ of sludge with a suspended solids concentration of 3.8 percent.  Recently they added a belt filter press (that follows settling and DAF).  The filter press produces a sludge with a solids concentration of 24 percent.  What annual volume savings of sludge have they achieved? \emph{(answer:about 10,000 m$^3$)}

\item \emph{(12 pt)} \textbf{Ecosystems, trophic states, and foodwebs}
Read Sections 5.3.2 and 5.3.3 (pp.186--190) and Section 5.6 (pp. 204--210) in the textbook on ecosystems and prepare a complete food chain diagram (starting with primary producers) that illustrates bio-concentration and bio-accumulation of mercury in humans that consume tuna. \textbf{less than 300 words}

\item \emph{(Extra Credit 10 pt)} \textbf{Write an Exam Question} Compose a challenging and thoughtful question on drinking water treatment for the next midterm exam.

\end{enumerate}
\end{document}