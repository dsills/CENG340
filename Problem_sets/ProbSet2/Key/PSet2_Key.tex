\documentclass[12pt,letterpaper]{article}
%\documentstyle[11pt]{article}
\usepackage[utf8]{inputenc}
\usepackage{amsmath}
\newcommand{\var}[1]{{\operatorname{#1}}}
%\usepackage{breqn}
\usepackage{xfrac}
\usepackage{amsfonts}
\usepackage{amssymb}
\usepackage[version = 3]{mhchem}
\usepackage{chemstyle}
%%For Table perhaps%%
%\usepackage{graphics}
\usepackage{graphicx}
\usepackage{epstopdf}
%\usepackage{tabularx,ragged2e,booktabs,caption}
%\newcolumntype{C}[1]{>{\Centering}m{#1}}
%\renewcommand\tabularxcolumn[1]{C{#1}}
\usepackage[left=2cm,right=2cm,top=2cm,bottom=2cm]{geometry}
\usepackage{subcaption} 
\usepackage{caption}
%\usepackage{siunitx}
%\usepackage{subfig}



\begin{document}
\setlength{\parindent}{0cm} 


\frenchspacing


% Default margins are too wide all the way around. I reset them here
%\setlength{\topmargin}{-.5in}
%\setlength{\textheight}{9in}
%\setlength{\oddsidemargin}{.125in}
%\setlength{\textwidth}{6.25in}




\title {Problem Set 2---\textbf{Key}} 
\author {CENG 340--Introduction to Environmental Engineering\\
Instructor: Deborah Sills}
\date {September 18, 2013}
\maketitle

\begin{enumerate}

\item Arsenic

[MCL]$\mathrm{_{arsenic}}$ = 10 ppb$_m$; MW$\mathrm{_{arsenic}}$ = 75 $\mathrm{\frac{g}{mole}}$

\begin{enumerate}
\item
\begin{equation*}
[As] = \mathrm{10 \, ppb_m \times \frac{1\, ppm_m}{10^{3}\, ppb_m} = 0.01\, ppm_m = 0.01 \, \frac{mg}{L}}
\end{equation*}\\

\begin{equation*}
[As] = \mathrm{0.01 \, \frac{mg}{L}\times \frac{1 \, mole}{75\, g}\, = 1.3\times 10^{-4}\, \frac {mmole}{L}}
\end{equation*}\\

\begin{equation*}
[As] = \mathrm{0.01 \, \frac{mg}{L}\times 10^3 \frac{\mu g}{mg} = 10\, \frac{\mu g}{L}}\\
\end{equation*}\\

\begin{equation*}
[As] = \mathrm{1.3\times 10^{-4}\, \frac {mmole}{L}\times 10^6\, \frac{nmole}{mmole} = 130\, \frac{nmole}{L} }
\end{equation*}\\

\item Exposure to arsenic may lead to skin damage, problems with circulatory systems, and increased risk of cancer.  (Source: water.epa.gov).\\

\item Naturally occurring arsenic has been found in water worldwide in every continent (Table 10.5 in the text book).\\

Also, (not required for PSet) arsenic in drinking water sources comes from orchard runoff, electronics and glass production. (source: water.epa.gov).\\

\end{enumerate}

\item Turbid water is opaque, hazy, and cloudy.  Small particles in water cause it to be turbid.  In other words, turbid water is dirty.\\

\item 
\begin{enumerate}
\item
TS = 45 + 30 + 100 + 10 + 20 + 25 = 230 mg/L

\item TDS = 45 + 30 + 10 + 20 = 105 mg/L

\item TSS = 100 + 25 = 125 mg/L

\item VSS = 25 mg/L

\item FDS = 45 + 30 + 10 = 85 mg/L\\

\end{enumerate}

\item 

[VC]$\mathrm{_{(aq)} = \frac{2\, mg}{60\, mL} = 0.05 \,\frac{g}{L}}$\\

MW$\mathrm{_{VC}}$ = 62.5 $\mathrm{\frac{g}{mole}}$\\

[VC]$\mathrm{_{(aq)} = 0.05 \,\frac{g}{L}\times \frac{1\, mole}{62.5\, g} = 8 \times10^{-4}\, \frac{mole}{L}}$\\

K$\mathrm{_{H} = 10^{0.04}\, \frac{\frac{mole_g}{L_g}}{\frac{mole_{aq}}{L_{aq}}}}$ (definition of dimensionless K$\mathrm{_H}$ found on p. 67 of the text book.\\

\begin{enumerate}
\item

\begin{align*}
\cee{VC_{(aq)} &<=>[K_H] VC_{(g)}} 
\end{align*}\\

Write the equilibrium relationship of aqueous and gaseous VC---make sure you write the equation in such a way that the units of K$\mathrm{_H}$ work out.\\

\begin{equation*}
\mathrm{K_H = \frac{VC_{(g)}}{VC_{(aq)}} = 10^{0.04}\, \frac{\frac{mole_g}{L_g}}{\frac{mole_{aq}}{L_{aq}}}}
\end{equation*}\\

Calculate the concentration of VC (moles/L) in the air from Henry's constant:\\

\begin{equation*}
\mathrm{VC_{(g)} = K_H\times VC_{(aq)} = 10^{0.04}\times 8\times10^{-4} = 8.8\times 10^{-4}\, \frac{mole}{L}\, gas}
\end{equation*}\\

Volume of the air = 100 m$^3$.  Use this and calculate total moles of VC in the air:\\

moles VC = $\mathrm{8.8\times 10^{-4}\, \frac{mole}{L}\, \times 100\, m^3\, \times \frac{1000\, L}{m^3}= 88\, moles\,\, VC\, in\, the\, air}$\\

Use the ideal gas law and calculate the Volume of VC in the air:\\

\begin{equation*}
\mathrm{V_{VC} = \frac{nRT}{P} = \frac{88\, moles \times 8.205\times 10^{-5}\, \frac{m^3atm}{moleK}\times 298.15\, K}{1\, atm} = 2.14\, m^3}
\end{equation*}\\

Calculate the volume fraction and multiply by $10^6$ to obtain the gaseous concentration of VC in units of ppm$\mathrm{_v}$\\

\begin{equation*}
\mathrm{[VC]_g = \frac{V_{VC}}{V_{tot}}\times 10^6 = \frac{2.1\,m^3}{100\, m^3}\times 10^6 = 21,500 \, ppm_v}
\end{equation*}\\

\begin{equation*}
\mathrm{21,500 \, ppm_v >> 10 \, ppm_v}
\end{equation*}\\

BAD NEWS!

\item The students should have put the open (or unsealed bottle) in the \emph{fume hood} as quickly as possible and told everyone to leave the lab immediately.\\
\end{enumerate}

\item (16 pts)
Assume atmospheric pressure of 1 atm, and convert atmospheric CO$_2$ concentration of 390 ppm$\mathrm{_v}$ to partial pressure of $\mathrm{CO_2}$\\

\begin{equation*}
\mathrm{[CO_{2}]_g = 390 ppm_v\times \frac{1\, atm}{10^6\, ppm_v} = 3.9\times 10^{-4}\, atm} 
\end{equation*}\\

$\mathrm{[CO_2]_{(g)}}$ values between 310 and 400 ppm$\mathrm{_v}$ acceptable.  Alternatively, P$\mathrm{_{CO_2} = 10^{-3.5}\, atm}$ is acceptable as well.\\

$\mathrm{K_H = 29.4\, \frac{L\times atm}{mole}}$\\

$\mathrm{MW_{CO_2} = 44\, \frac{g}{mole}}$\\

\begin{enumerate}
\item Step 1: Write equilibrium relationship for partitioning of $\mathrm{CO_2}$ from the gas to the aqueous phase.  Write the reaction in such a way that the units of $K_H$ work out.

\begin{align*}
\cee{CO_{2(aq)} &<=>[K_H] CO_{2(g)}} 
\end{align*}\\

\item Calculate the dissolved aqueous concentration of $\mathrm{CO_2}$ using Henry's Law constant and the atmospheric concentration of $\mathrm{CO_2}$, which you can find online.\\

\begin{equation*}
\mathrm{K_H = \frac{[CO_{2}]_{(g)}}{[CO_{2}]_{(aq)}}} 
\end{equation*}\\

\begin{equation*}
\mathrm{[CO_{2}]_{(aq)} = \frac{[CO_{2}]_{(g)}}{K_H} = \frac{3.9\times 10^{-4}\, atm}{29.4\, \frac{L\times atm}{mole}} = 1.33\times 10^{-5}\, \frac{moles}{L}} 
\end{equation*}\\


\begin{equation*}
\mathrm{[CO_{2}]_{(aq)} = 1.33\, \frac{moles}{L}\times 44\, \frac{g}{mole}\times \frac{1000\, mg}{g} = 0.58\, \frac{mg}{L}} 
\end{equation*}\\

\item When the concentration of dissolved aqueous $\mathrm{CO_2}$ increases, the concentration of $\mathrm{H^+}$ also increases and pH decreases.

\end{enumerate}

\item 

Balance the reaction:\\

\begin{align*}
\cee{\mathrm{C_2H_5OH}}\,  + \,  3\mathrm{O_2}\, \rightarrow \, 2\mathrm{CO_2} \, + \,  3\mathrm{H_2O}
\end{align*}\\


Convert lbs of ethanol to kg of ethanol:\\

\begin{equation*}
\mathrm{500\, lbs\, ethanol \times \frac{1\, kg}{2.2\, lbs} = 227\, kg\, ethanol}
\end{equation*}\\

\begin{enumerate}

\item Use the mass ratios noted above below the equation to calculate the mass of O$_2$ required for biodegradation:\\

\begin{equation*}
\mathrm{Mass\, of\, O_2 = 227\, kg\, eth\times \frac{96\, g\, O_2}{46\, g\, eth} = 474\, kg\, O_2}
\end{equation*}\\



\item Calculate the mass of CO$_2$ produced:\\

\begin{equation*}
\mathrm{Mass\, of\, CO_2 = 227\, kg\, eth\times \frac{88\, g\, O_2}{46\, g\, eth} = 435\, kg\, CO_2}
\end{equation*}\\

\item Convert kg of CO$_2$ to moles.  Then use the ideal gas law to convert to volume of CO$_2$.\\

\begin{equation*}
\mathrm{Moles\, of\, CO_2 = 435\, kg\, CO_2\times \frac{1000\, g}{kg} \times \frac{1\, mole}{44\, g} = 9881\, mole}
\end{equation*}\\

\begin{equation*}
\mathrm{V = \frac{nRT}{P} = \frac{9881\times 8.205\times 10^{-5}\, \frac{m^3atm}{moleK}\times 303.15\, K}{1\, atm} = 245 \, m^3}
\end{equation*}\\

\end{enumerate}

\item
$\mathrm{H_2SO_4}$ is a strong acid:\\

\begin{align*}
\cee{H_2SO_4\, \rightarrow \, 2H^+ \, + \,  SO_4^{2-}}
\end{align*}\\



\begin{enumerate}
\item To calculate moles of [H$^+$], calculate number of moles of $\mathrm{H_2SO_4}$ and multiply by 2:\\


\begin{equation*}
\mathrm{10\, \frac{mg\, H_2SO_4}{L}\times \frac{1\, g}{1000\, mg}\times \frac{1\,moles}{98\, g}\times \frac{2 \, mole \, H^+}{mole\, H_2SO_4} = 2\times 10^{-4}\,\frac{mole}{L}\, H^+}
\end{equation*}\\


\begin{equation*}
\mathrm{pH = -log[H^+] = -log(2\times 10^{-4}) = 3.7}
\end{equation*}\\


\item Normality of H$_2$SO$_4$ solution equal the moles per Liter of H$_+$, which equals $\mathrm{2\times 10^{-4}\,\frac{eq}{L}}$\\

\item 
Convert [HCO$_3^-$] from mg/L as CaCO$_3$ to eq/L:\\

\begin{equation*}
\mathrm{[HCO_3^-] = 90\, \frac{mg\, CaCO_3}{L}\times \frac{1\,g}{1000\, mg}\times \frac{1\, eq}{50\, g} = 1.8\times 10^{-3}\, \frac{eq}{L}}
\end{equation*}\\

Use:\\

\begin{equation*}
\mathrm{C_{H^+}\times V_{H^+\, solution} = C_{bicarb} \times V_{bicarb\, solution}}
\end{equation*}\\

\begin{equation*}
\mathrm{V_{bicarb\, solution} = \frac{C_{H^+}\times V_{H^+\, solution}}{C_{bicarb}} = \frac{2\times 10^{-4}\, \frac{eq}{L}\times 100\, mL}{1.8\times 10^{-3}}\, \frac{eq}{L} = 11\, mL}
\end{equation*}\\
\end{enumerate}

\item
First Order Reaction:\\

\begin{equation*}
\mathrm{A = A_0\times e^{-kt}}
\end{equation*}\\

\begin{enumerate}

\item 90\% of A destroyed, so 10\% of initial concentration remains.\\

\begin{equation*}
\mathrm{0.1A_0 = A_0\times e^{-0.1t}}
\end{equation*}\\

\begin{equation*}
\mathrm{t = \frac{ln(\frac{0.1}{1})}{-0.1\, day^{-1}}  = 23\, days}
\end{equation*}\\

\item Same as part (a) except that 99\% destroyed, so 1\% remaining:

\begin{equation*}
\mathrm{t = \frac{ln(\frac{0.01}{1})}{-0.1\, day^{-1}}  = 46\, days}
\end{equation*}\\

\item Same as part (a) except that 99.9\% destroyed, so 0.1\% remaining:

\begin{equation*}
\mathrm{t = \frac{ln(\frac{0.001}{1})}{-0.1\, day^{-1}}  = 69\, days}
\end{equation*}\\

\end{enumerate}
















\end{enumerate}
\end{document}