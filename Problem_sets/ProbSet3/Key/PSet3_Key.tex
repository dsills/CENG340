\documentclass[12pt,letterpaper]{article}
%\documentstyle[11pt]{article}
\usepackage[utf8]{inputenc}
\usepackage{amsmath}
\usepackage{xfrac}
\usepackage{amsfonts}
\usepackage{amssymb}
\usepackage[version = 3]{mhchem}
\usepackage{chemstyle}
%%For Table perhaps%%
%\usepackage{graphics}
\usepackage{graphicx}
\usepackage{epstopdf}
\usepackage{tabularx,ragged2e,booktabs,caption}
%\newcolumntype{C}[1]{>{\Centering}m{#1}}
\renewcommand\tabularxcolumn[1]{C{#1}}
\usepackage[left=2cm,right=2cm,top=2cm,bottom=2cm]{geometry}
\usepackage{subcaption} 
\usepackage{caption}
\usepackage[colorlinks]{hyperref}
\usepackage[svgnames]{xcolor}
\hypersetup{citecolor=DeepPink4}
\hypersetup{linkcolor=DarkRed}
\hypersetup{urlcolor=DarkBlue}
\usepackage{cleveref}
\usepackage{enumerate}

\begin{document}
\setlength{\parindent}{0cm} 


\frenchspacing

\title {Problem Set 3---Key} 
\author {CENG 340--Introduction to Environmental Engineering\\
Instructor: Deborah Sills}
\date {September 17, 2013}
\maketitle



\section *{Questions}
\begin{enumerate}
\item Precipitation--Dissolution

\begin{align*}
\cee{CaCO_3 &<=>[K_{sp}] [Ca^{2+}] + [CO_3^{2-}]} 
\end{align*}

\begin{equation*}
\mathrm{K_{sp} = [Ca^{2+}][CO_3^{2-}]= 10^{-8.3}}
\end{equation*}

Since equilibrium occurs at the \textbf{end point of a reaction}, we need to calculate the final concenrations of Ca$^{2+}$ and Ca$^{2-}$. To do so, set up a RICE table.  R(reaction), I(initial conditions), C(change), and E(equilibrium).  But first need to convert initial concentrations from mg/L to mole/L:

\begin{equation*}
\mathrm{ [Ca^{2+}] = 50\, \frac{mg}{L}\, \times \frac{1\, g}{1000\, mg}\times \frac{1\, mole}{40\, g} = 8.3\times 10^{-4}\, \frac{moles}{L}}
\end{equation*}

\begin{equation*}
\mathrm{ [CaO_3^{2-}] = 50\, \frac{mg}{L}\, \times \frac{1\, g}{1000\, mg}\times \frac{1\, mole}{60\, g} = 1.25\times 10^{-3}\, \frac{moles}{L}}
\end{equation*}\\



\begin{minipage}{\linewidth}
\centering
\captionof{table}{Rice Diagram} \label{tab:title}

\begin{tabular}{|c|c|c|p{3.1cm}|}\toprule[1.25pt]
\bf {R}eaction & \bf Ca$^{2+}$	& \bf CO$_3^{2-}$	& \bf CaCO$_3$\\\midrule
\bf{I}nitial	&  $\mathrm{8.3\times 10^{-4}\, \frac{mole}{L}}$	& $\mathrm{1.25\times 10^{-3}\, \frac{mole}{L}}$ & \\ \hline
\bf{C}hange	& -x 	& -x	& \\ \hline
\bf{E}quilibrium & $\mathrm{(8.3\times 10^{-4} - x)\, \frac{mole}{L}}$	& $\mathrm{(1.25\times 10^{-3} -x)\, \frac{mole}{L}}$	&  \\  \hline

\bottomrule[1.25pt]

\end {tabular}\par
\end{minipage}\\\\

Now we can substitute the equilibrium concentrations into the above equation for K$\mathrm{_{sp}}$

\begin{equation*}
\mathrm{K_{sp} = [Ca^{2+}][CO_3^{2-}]= (8.3\times 10^{-4} - x)\times(1.25\times 10^{-3} -x) = 10^{-8.3}}
\end{equation*}\\

Solve for x. \\

x = $\mathrm{8.2 \times 10^{-4}\, \frac{mole}{L}}$\\

[Ca$\mathrm{^{2+}]_{equilibrium} = 1.25\times 10^{-3} - 8.2 \times 10^{-4} = 4.2 \times 10^{-4}\, \frac{mole}{L}}$\\

[Ca$\mathrm{^{2+}]_{equilibrium}  = 4.2 \times 10^{-4}\, \frac{mole}{L}\times \frac{40\, g}{mole}\times \frac{1000\, mg}{g} = 17\, \frac{mg}{L}}$\\

\vspace{0.2in}

\item  Precipitation--Dissolution of Iron Hydroxide\\

\begin{align*}
\cee{FeOH_3 &<=>[K_{sp}] [Fe^{3+}] + [OH^{-}]} 
\end{align*}

pK$_{sp} = 38.57$

\begin{equation*}
\mathrm{K_{sp} = [Fe^{3+}][OH^{-}]^3=  10^{-38.57}}
\end{equation*}\\

Equilibrium measures end-point of reaction, so need to use final iron concentration:\\

$\mathrm{[Fe^{3+}] = 0.2\, \frac{mg}{L}}$\\

\begin{equation*}
\mathrm{[Fe^{3+}] = 0.2\, \frac{mg}{L}\times \frac{1\, g}{1000\, mg} \times \frac{1\, mole}{56\, g} = 3.6 \times 10^{-6}\, \frac{mole}{L}}
\end{equation*}

\begin{equation*}
\mathrm{[OH^-] = (\frac{K_{sp}}{[Fe^{3+}]})^{\sfrac{1}{3}} = (\frac{10^{-38.57}}{3.6\time 10^{-6}})^{\sfrac{1}{3}} = 9.1\times 10^{-12}\, \frac{mole}{L}} 
\end{equation*}

\begin{equation*}
\mathrm{[H^+] = \frac{10^{-14}}{[OH^-]} = \frac{10^{-14}}{9.1\times 10^{-12}} = 0.001}
\end{equation*}

\begin{equation*}
\mathrm{pH = -log(H^+) = 2.96}
\end{equation*}

\vspace{0.2in}

\item Calculate alkalinity in mg/L as CaCO$_3$:\\

\begin{equation*}
\mathrm{[HCO_3^{-}] = 111 \frac{mg}{L}\times \frac{1\, mole}{61\, g}\times \frac{1\, g}{1000\, mg}\times \frac{1\, eq}{mole} = 1.8\times 10^{-3}\, \frac{eq}{L}\, of \, alkalinity}
\end{equation*}

\begin{equation*}
\mathrm{[CO_3^{2-}] = 17 \frac{mg}{L}\times \frac{1\, mole}{60\, g}\times \frac{1\, g}{1000\, mg}\times \frac{1\, eq}{mole} = 5.7\times 10^{-4}\, \frac{eq}{L}\, of \, alkalinity}
\end{equation*}\\

Approximate Alkalinity (eq/L) = $\mathrm{1.8\times 10^{-3} + 5.7\times 10^{-4}\ = 2.4\times 10^{-3}\, \frac{eq}{L}}$\\

Approximate Alkalinity (eq/L) = $\mathrm{2.4\times 10^{-3}\, \frac{eq}{L} \times \frac{100\, g\, CaCO_3}{mole}\times \frac{1\, mole}{2\, eq} = 119\, \frac{mg}{L}}$\\

\vspace{0.2in}

\item Acid-Base Equilibrium for HOCl in drinking water.\\
Calculate fraction of HOCl that is in the dissociatead form:

\begin{align*}
\cee{[HOCl]_{aq} &<=>[K_{a}] [H^{+}]_{aq} + [OCl^{-}]_{aq}} 
\end{align*}\\

pka = 7.5

$\mathrm{K_a} = 10^{-7.5}$

Since it's drinking water, assume pH = 7.

$\mathrm{H^+} = 10^{-7}$

\begin{equation*}
\mathrm{K_a = \frac{[H^+][OCl^-]}{[HOCl]} = 10^{-7.5}}
\end{equation*}\\

Rearrange equation, and substitute values for K$_a$ and H$^+$:\\

\begin{equation*}
\mathrm{\frac{[OCl^-]}{[HOCl]} = \frac{10^{-7.5}}{10^{-7}}}
\end{equation*}\\

\begin{equation*}
\mathrm{[OCl^-] = 10^{-0.5}\times [HOCl] = 0.316[HOCl]}
\end{equation*}\\

\begin{equation*}
\mathrm{Fraction Dissociated = \frac{[HOCl]}{[HOCl] + [OCl^-} = \frac{[HOCl]}{[HOCl] + 0.316[HOCl]} = \frac{1}{1+ 0.316} = 0.76}
\end{equation*}


\vspace{0.2in}

\item Atrazine




\item Oil Spill nineteen years ago:\\

$\mathrm{C_0 = 400\, \frac{mg}{L}}$
$\mathrm{C \, after\, 19 y = 400\, \frac{mg}{L}}$
\begin{enumerate}

\item Try a zero order rate equation:
\begin{equation*}
\mathrm{\frac{dC}{dt} = -k}
\end{equation*}\\

After integration:\\

\begin{equation*}
\mathrm{C = C_0 -kt}
\end{equation*}

where C$_0$ = 400 $\mathrm{\frac{mg}{g}}$\\
C = 20 $\mathrm{\frac{mg}{g}}$\\
t = 19 year\\

Solve for k = 20 t$_{-1}$, and substitute k into the integrated zero rate equation above to obtain\\

\begin{equation*}
\mathrm{C = C_0 -20\, year^{-1}\times 20 \, year = 0}
\end{equation*}\\



Answer: Yes the engineer is correct if the degradation rate is zero order.

\item To find the ``worst-case scenario,'' calculate the concentration of the pollutant after twenty years using a first order and second order rate equation.\\

\textbf{First Order:}\\

\begin{equation*}
\mathrm{C = C_0\times e^{-kt}}
\end{equation*}\\

Solve for k:\\

\begin{equation*}
\mathrm{k = -\frac{ln\frac{C}{C_0}}{t} = -\frac{ln\frac{20}{400}}{19\, y} = 0.16 \, y^{-1}}
\end{equation*}\\

Use k and solve for the time it will take to for C = 1 $\mathrm{\frac{mg}{kg}}$, assuming first-order kinetics:\\

\begin{equation*}
\mathrm{t = -\frac{ln\frac{C}{C_0}}{k} = -\frac{ln\frac{1}{400}}{0.16\, year^{-1}} = 37 \, y}
\end{equation*}\\

\textbf{Second Order:}\\

\begin{equation*}
\mathrm{C = \frac{C_0}{1 + C_0kt}}
\end{equation*}\\

Rearrange and solve for k:

\begin{equation*}
\mathrm{k = \frac{\frac{1}{C}-\frac{1}{C_0}}{t} = \frac{(\frac{1}{20}-\frac{1}{400})\, \frac{kg}{mg}}{19\, y} = 0.003\, \frac{kg}{mg\times y}}
\end{equation*}\\

Use k and solve for the time it will take for C = 1 $\mathrm{\frac{mg}{kg}}$, assuming second-order kinetics:\\

\begin{equation*}
\mathrm{t = \frac{\frac{1}{C}-\frac{1}{C_0}}{k} = \frac{(\frac{1}{1}-\frac{1}{400})\, \frac{kg}{mg}}{0.003\, \frac{kg}{mg\times y}} = 333\, y}
\end{equation*}\\

In conclusion the ``worst-case scenario'' is second order, in which case, it would take 333 y for the pollutant to degrade.  However, first order is more likely.\\
\end{enumerate}
\vspace{0.2in}
\item Landfill
\vspace{0.2in}
\item Zamboni








\end{enumerate}
\end{document}
