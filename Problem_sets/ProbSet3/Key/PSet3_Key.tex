\documentclass[12pt,letterpaper]{article}
%\documentstyle[11pt]{article}
\usepackage[utf8]{inputenc}
\usepackage{amsmath}
\usepackage{xfrac}
\usepackage{amsfonts}
\usepackage{amssymb}
\usepackage[version = 3]{mhchem}
\usepackage{chemstyle}
%%For Table perhaps%%
%\usepackage{graphics}
\usepackage{graphicx}
\usepackage{epstopdf}
\usepackage{tabularx,ragged2e,booktabs,caption}
%\newcolumntype{C}[1]{>{\Centering}m{#1}}
\renewcommand\tabularxcolumn[1]{C{#1}}
\usepackage[left=2cm,right=2cm,top=2cm,bottom=2cm]{geometry}
\usepackage{subcaption} 
\usepackage{caption}
\usepackage[colorlinks]{hyperref}
\usepackage[svgnames]{xcolor}
\hypersetup{citecolor=DeepPink4}
\hypersetup{linkcolor=DarkRed}
\hypersetup{urlcolor=DarkBlue}
\usepackage{cleveref}
\usepackage{enumerate}

\begin{document}
\setlength{\parindent}{0cm} 


\frenchspacing

\title {Problem Set 3---Key} 
\author {CENG 340--Introduction to Environmental Engineering\\
Instructor: Deborah Sills}
\date {September 17, 2013}
\maketitle

\section *{Due Date}
Wednesday 25 September, in class.
\section *{Learning Goals}
\begin{enumerate}
\item Write and apply equilibrium equations to acid-base, precipitation-dissolution, and and sorption reactions.
\item Apply kinetic equations to determine reaction times.
\item Use the law of conservation of mass to write and apply a mass balance expression.
\end{enumerate}


\section *{Questions}
\begin{enumerate}
\item 
$\mathrm{C_0 = 400\, \frac{mg}{L}}$
$\mathrm{C \, after\, 19 y = 400\, \frac{mg}{L}}$
\begin{enumerate}

\item Try a zero order rate equation:
\begin{equation*}
\mathrm{\frac{dC}{dt} = -k}
\end{equation*}\\

After integration:\\

\begin{equation*}
\mathrm{C = C_0 -kt}
\end{equation*}

where C$_0$ = 400 $\mathrm{\frac{mg}{g}}$\\
C = 20 $\mathrm{\frac{mg}{g}}$\\
t = 19 year\\

Solve for k = 20 t$_{-1}$, and substitute k into the integrated zero rate equation above to obtain\\

\begin{equation*}
\mathrm{C = C_0 -20\, year^{-1}\times 20 \, year = 0}
\end{equation*}\\

Answer: Yes the engineer is correct if the degradation rate is zero order.

\item To find the ``worst-case scenario,'' calculate the concentration of the pollutant after twenty years using a first order and second order rate equation.\\

\textbf{First Order:}\\

\begin{equation*}
\mathrm{C = C_0\times e^{-kt}}
\end{equation*}\\

Solve for k:\\

\begin{equation*}
\mathrm{k = -\frac{ln\frac{C}{C_0}}{t} = -\frac{ln\frac{20}{400}}{19\, y} = 0.16 \, y^{-1}}
\end{equation*}\\

Use k and solve for the time it will take to for C = 1 $\mathrm{\frac{mg}{kg}}$, assuming first-order kinetics:\\

\begin{equation*}
\mathrm{t = -\frac{ln\frac{C}{C_0}}{k} = -\frac{ln\frac{1}{400}}{0.16\, year^{-1}} = 37 \, y}
\end{equation*}\\

\textbf{Second Order:}\\

\begin{equation*}
\mathrm{C = \frac{C_0}{1 + C_0kt}}
\end{equation*}\\

Rearrange and solve for k:

\begin{equation*}
\mathrm{k = \frac{\frac{1}{C}-\frac{1}{C_0}}{t} = \frac{(\frac{1}{20}-\frac{1}{400})\, \frac{kg}{mg}}{19\, y} = 0.003\, \frac{kg}{mg\times y}}
\end{equation*}\\

Use k and solve for the time it will take for C = 1 $\mathrm{\frac{mg}{kg}}$, assuming second-order kinetics:\\

\begin{equation*}
\mathrm{t = \frac{\frac{1}{C}-\frac{1}{C_0}}{k} = \frac{(\frac{1}{1}-\frac{1}{400})\, \frac{kg}{mg}}{0.003\, \frac{kg}{mg\times y}} = 333\, y}
\end{equation*}\\

In conclusion the ``worst-case scenario'' is second order, in which case, it would take 333 y for the pollutant to degrade.  However, first order is more likely.






\end{enumerate}
\item blah blah
\end{enumerate}
\end{document}
