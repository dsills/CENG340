\documentclass[12pt,letterpaper]{article}
%\documentstyle[11pt]{article}
\usepackage[utf8]{inputenc}
\usepackage{amsmath}
\usepackage{xfrac}
\usepackage{amsfonts}
\usepackage{amssymb}
\usepackage[version = 3]{mhchem}
\usepackage{chemstyle}
%%For Table perhaps%%
%\usepackage{graphics}
\usepackage{graphicx}
\usepackage{epstopdf}
\usepackage{tabularx,ragged2e,booktabs,caption}
\newcolumntype{C}[1]{>{\Centering}m{#1}}
\renewcommand\tabularxcolumn[1]{C{#1}}
\usepackage[left=2cm,right=2cm,top=2cm,bottom=2cm]{geometry}
\usepackage{subcaption} 
\usepackage{caption}
\usepackage[colorlinks]{hyperref}
\usepackage[svgnames]{xcolor}
\hypersetup{citecolor=DeepPink4}
\hypersetup{linkcolor=DarkRed}
\hypersetup{urlcolor=DarkBlue}
\usepackage{cleveref}
\usepackage{enumerate}

\begin{document}
\setlength{\parindent}{0cm} 


\frenchspacing


\title {\Large{\textbf{Problem Set 10}}\\ \large{CENG 340--Introduction to Environmental Engineering\\
Instructor: Deborah Sills\\ \textbf{December 2, 2013}}}

\author {}
\date {}
\maketitle

\vspace{-1in}
\section *{Due Date and Grading}
Monday, December 9th, by 5pm.\



\section *{Learning Objectives}
\begin{enumerate}
\item Be familiar with the basic design options for ecological sanitation (ECOSAN) latrines.
\item Integrate mass balances with the Monod Model (for microbial growth kinetics) to develop design equations for biological wastewater treatment reactors.
\item Relate solids retention time (SRT), food-to-microorganism ratio, sludge wasting, and microbial growth kinetics to the design and operation of biological treatment reactors.
\end{enumerate}

\section *{Relevant Reading}

\begin{enumerate}
\item Ecosan Technical Handbook (link included as part of Problem 1):\\ http://www.wateraid.org/~/media/Publications/construction-ecological-sanitation-latrine-technical-handbook.pdf

\item Textbook: pp. 475--489
\end{enumerate}



 

\section *{Problems}
\begin{enumerate}

\item \emph{(20 pts)} \textbf{Ecological Sanitaion (Ecosan) Latrines}:
Read the following \href{http://www.wateraid.org/~/media/Publications/construction-ecological-sanitation-latrine-technical-handbook.pdf}{Technical Handbook} on construction of Ecosan Latrines published by \emph{Water Aid}, and propose a design recommendation to an under-served community.  Your design recommendation should include three sections:

\begin{enumerate}
\item A \emph{brief} introduction (no more than one paragraph) to Ecosan and why you recommend this technology.
\item A \emph{brief} (no more than one paragraph) rationale as to why you recommend a specific Ecosan system---one of the urine diversion systems (single or double vault), or a composting latrine (may include urine diversion or not; may be single or dual chamber).
\item A \emph{brief} (no more than one paragraph) description of the latrine design and how the latrine works.
\item A \emph{brief} (no more than one paragraph) list of operating and maintenance procedures required for the latrine to function properly, and how you propose to ensure the latrine is maintained appropriately.
\end{enumerate}




\item \emph{(14 pts)} \textbf{Microbial Growth Kinetics---Batch Reactor}\\

Assume:\\

A batch reactor with Q = 0\\

%k = 10 $\mathrm{\frac{mg \, BOD_u}{mg\, X\times day}}$\\
%
%Y = 0.65 $\mathrm{\frac{mg \, X}{mg\, BOD_u}}$\\

$\mu_{max}$ = 6.5 day$^{-1}$\\

K$_s$ = 35 $\mathrm{\frac{mg \, BOD_u}{L}}$\\

k$_d$ = 0.10 day$^{-1}$
 
\begin{enumerate}
\item If the food supply is unlimited and large (promoting exponential growth), and the initial concentration of bacteria is 0.25 $\mathrm{\frac{mg \,X}{L}}$, then what will be the biomass concentration, X, (in mg cells/L) at the end of three days? \emph{Answer: 5.5$\times 10^7\,\, \frac{mg\, cells}{L}$}
 
\item Bacteria multiply by binary fission, doubling their number with each new generation. Calculate the ``generation time'' (in days) for this system (i.e., the length of time it takes for the population to double). \emph{Answer: 2.6 hours}

\end{enumerate}

\item \emph{(14 pts)} \textbf{Microbial Growth Kinetics---CMFR}\\
Assume that the BOD removal rate can be described by the following equation that is first order with respect to both cell concentration (X) and BOD concentration (S):

\begin{align}
\mathrm{-\frac{dS}{dt} = kXS}
\end{align}
where: k = a first-order, growth-rate constant (day$^{-1}$). Also, assume that cell growth rate is directly proportional to the rate of BOD removal: 

\begin{align}
\mathrm{\frac{dX}{dt} = -Y\frac{dS}{dt}}
\end{align}

where: Y = yield coefficient $\mathrm{\left(\frac{mg \,\, cells}{mg \, \, BOD \, \, consumed}\right)}$. And assume that cell death is first order with respect to cell concentration: 


\begin{align}
\mathrm{\frac{dX}{dt} = -k_dX}
\end{align}


where: k$_d$ = a first-order, decay-rate constant (day$^{-1}$).\\

Derive an equation to predict the steady state effluent BOD concentration (S) from a CMFR if: S$_0$ = influent BOD concentration (mg/L); Q = influent flow rate (L/day);
V = CMFR reactor volume (L); X$_0$= influent cell concentration = zero; and the CMFR has no recycle flow.

\item \emph{(14 pts)}
\begin{enumerate}

\item A waste is to be treated aerobically in a CMFR with no recycle. Determine the critical SRT ($\theta_{cmin}$) using the following constants:\\

$\mu_{max}$ = 4.2 day$^{-1}$\\
 
K$_s$ = 40 mg/L\\

k$_d$ = 0.1 day$^{-1}$\\
 
The input waste concentration, S$_0$, is 200 mg/L BOD$_u$.\\
\emph{Answer: 7.1 hours}
 
\item Assuming that the design SRT ($\theta_c$ value) must be at least 20$\times \theta_{cmin}$ to provide a safety factor against wash out of cells, calculate the effluent substrate concentration, S.  \emph{Answer: 2.75 mg/L}
\end{enumerate}

\item \emph{(10 pts)} Textbook: 11.6 \emph{Answer: 656 kg/day}
\item \emph{(14 pts)}Textbook: 11.7  \emph{Answers: a: $4.2\times 10^6$ L; b: 2.9 hour; c: 4515 kg/day; e: 0.97 $\mathrm{\frac{lb\, BOD_5}{lb\, MLVSS\times day}}$; f: 4 days}

\item \emph{(14 pts)} Textbook: 11.16.  \textbf{Assume that MLVSS = 0.6 of MLSS.}\\
\emph{Answer: 0.33 $\mathrm{\frac{lb\, BOD_5}{lb\, MLVSS\times day}}$}

\end{enumerate}
\end{document}