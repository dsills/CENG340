\documentclass[12pt,letterpaper]{article}
%\documentstyle[11pt]{article}
\usepackage[utf8]{inputenc}
\usepackage{amsmath}
\usepackage{xfrac}
\usepackage{amsfonts}
\usepackage{amssymb}
\usepackage[version = 3]{mhchem}
\usepackage{chemstyle}
%%For Table perhaps%%
%\usepackage{graphics}
\usepackage{graphicx}
\usepackage{epstopdf}
\usepackage{tabularx,ragged2e,booktabs,caption}
%\newcolumntype{C}[1]{>{\Centering}m{#1}}
\renewcommand\tabularxcolumn[1]{C{#1}}
\usepackage[left=2cm,right=2cm,top=1cm,bottom=1cm]{geometry}
\usepackage{subcaption} 
\usepackage{caption}
\usepackage[colorlinks]{hyperref}
\usepackage[svgnames]{xcolor}
\hypersetup{citecolor=DeepPink4}
\hypersetup{linkcolor=DarkRed}
\hypersetup{urlcolor=DarkBlue}
\usepackage{cleveref}
\usepackage{enumerate}

\begin{document}
\setlength{\parindent}{0cm} 


\frenchspacing


\title {\Large{\textbf{Problem Set 5}}\\ \large{CENG 340--Introduction to Environmental Engineering\\
Instructor: Deborah Sills\\ \textbf{October 16, 2013}}}

\author {}
\date {}
\maketitle

\vspace{-1in}
\section *{Due Date}
Wednesday, 23 October, by 5pm.  Bring assignments to my office, or bring them to class on Wednesday morning.  I'll leave an envelope taped to my door, in case I'm not in my office.

\section *{Learning Goals}
\begin{enumerate}
\item Reflect on your learning in this class.
\item Size water treatment unit processes used for coagulation.
\item Apply principles of water chemistry to calculate alkalinity requirements for coagulation.
\item Apply principles of water chemistry to design water softening treatments.
\end{enumerate}

\section *{Relevant Sections in the Book}
10.5, 10.6, 10.7


\section *{Self Reflection \emph{(30 pts)}}

We are about halfway into the course, which is a good time for you to look back and reflect on your performance.  This allows you to see what you have learned and to identify places where you still have questions.  In addition, it provides me with an opportunity to answer your questions. 

\emph{This self reflection exercise will be graded on completeness and the quality of your response. Please type your answers and submit as a separate document with the problem set.  I will grade the reflection and a grader will grade the problems.} 

\begin{enumerate}
\item \emph{Overall assessment}:

On a scale of 0 to 100 what grade would you give yourself on

\begin{enumerate}
\item problem sets
\item lab reports
\item blog posts
\item quizzes
\item midterm exam
\item course overall
\end{enumerate}

\textbf{Explain your answers.}

\item \emph{Problem Solving and Lab Reports}:
\begin{enumerate}
\item What was the most serious problem that you had completing problem sets and lab reports? How did you deal with this problem?  How will you avoid or minimize this problem in the future?

\item What, if anything, did you learn about problem solving from the homework assignments?  If you learned something positive, what was it?  If you did not, why do you think homework did not help develop problem solving skills and what would be helpful.  

\item What, if anything, did you learn from writing lab reports and preparing high quality figures?  If you learned something positive, what was it?  If you did not, why do you think lab assignments did not help develop skills needed to effectively communicate results from laboratory and model fitting studies, and what would be helpful.    
\end{enumerate}

\item \emph{Preparation for Quizzes and the Mid-term exam}:
\begin{enumerate}
\item What was the most serious problem you had in preparing for quizzes and exams.  How did you deal with this problem?  How will you avoid or minimize this problem in the future?
\end{enumerate}
\end{enumerate}

\section *{Problems (70 pts)}
\begin{enumerate}
\item \emph{(12 pts)} \textbf{Iron Oxidation}--Back to CMFR vs. PFR:\\
Textbook: Problem 10.7 (assume that ``pseudo first-order rate constant'' is simply a first-order rate constant)
\emph{Answers: 4,286 m$^3$, 2.6 h; 528 m$^3$, 19 min}

\item \emph{(12 pts)} \textbf{Alkalinity requirements during coagulation:}\\
Textbook: Problem 10.3 \emph{Answer: 6.25 mg/L, 862,300 kg/y}

\item  \textbf{Water softening} Refer to example 10.3 on pp. 420-422. 
\begin{enumerate}

\item \emph{(10 pts)} Textbook: Problem 10.4 \emph{Answers: 250 mg/L, 89 mg/L, 339 mg/L}

\item \emph{(12 pts)} Textbook: Problem 10.5 \emph{Answers: 197.5 mg/L, 61.5 mg/L}

\item \emph{(12 pts)} \textbf{FE Formatted Question}\\Estimate the amount of lime, in tons/d, required to soften 5 MGD of water to the practical solubility limits.  The constituent concentrations are as follows: CO$_2$ (H$_2$CO$_3^*$) = 0.44 meq/L, Ca$^{2+}$ = 4.76 meq/L, Mg$^{2+}$ = 1.11 meq/L, Alkalinity = 3.96 meq/L, Cl$^-$ = 1.91 meq/L, SO$_4^{2-}$ = 1.58 meq/L.

\begin{enumerate}
\item 1.15 tons/d
\item 2.92 tons/d
\item 6.4 tons/d
\item 3.2 tons/d
\end{enumerate}
Show your work even though you wouldn't have to for the FE \emph{Answer: 2.92 tons/d}.



\end{enumerate}


  


\item \emph{(12 pts)} \textbf{Sizing a coagulation basin:}\\
Textbook: Problem 10.8.  \textbf{After you solve the problem as written, solve it again for water at a temperature of 24 $^0$C.} 

 













\end{enumerate}
\end{document}