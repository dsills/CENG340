\documentclass[12pt,letterpaper]{article}
%\documentstyle[11pt]{article}
\usepackage[utf8]{inputenc}
\usepackage{amsmath}
\newcommand{\var}[1]{{\operatorname{#1}}}
%\usepackage{breqn}
\usepackage{xfrac}
\usepackage{amsfonts}
\usepackage{amssymb}
\usepackage[version = 3]{mhchem}
\usepackage{chemstyle}
%%For Table perhaps%%
%\usepackage{graphics}
\usepackage{graphicx}
\usepackage{epstopdf}
%\usepackage{tabularx,ragged2e,booktabs,caption}
%\newcolumntype{C}[1]{>{\Centering}m{#1}}
%\renewcommand\tabularxcolumn[1]{C{#1}}
\usepackage[left=2cm,right=2cm,top=2cm,bottom=2cm]{geometry}
\usepackage{subcaption} 
\usepackage{caption}
%\usepackage{siunitx}
%\usepackage{subfig}



\begin{document}
\setlength{\parindent}{0cm} 


\frenchspacing


% Default margins are too wide all the way around. I reset them here
%\setlength{\topmargin}{-.5in}
%\setlength{\textheight}{9in}
%\setlength{\oddsidemargin}{.125in}
%\setlength{\textwidth}{6.25in}




\title {Problem Set 1---\textbf{Key}} 
\author {CENG 340--Introduction to Environmental Engineering\\
Instructor: Deborah Sills}
\date {September 1, 2013}
\maketitle

\begin{enumerate}

\item (40 pts)
Many questions about Floyd|Snyder or Dr. Heavner's career were acceptable.  Questions that were not specific (e.g., What was the most interesting project you worked on) received half credit.

\item (21 points)
\begin{enumerate}
\item Susquehanna River, White Deer Creek, and Spruce River Reservoir.
\item No violations.  Coliforms and lead (either or both) may be a problem.
\item Coliform is a biological constituent; lead is a chemical constituent.
\end{enumerate}

\item (39 points)
\begin{enumerate}
\item $\mathrm{\rho_{H_2O}} = 1\, \frac{g}{mL} = 1000\, \frac{g}{L}$
\begin{equation*}
\mathrm{[C_6H_6]_{MCL} = 0.005\, \frac{\var{mg}}{L}\times \frac{1 \, g}{1000 \, mg}\times \frac{1 \, L}{1000 \, g} = 5 \times 10^{-9}\,\frac{g}{g} \times 10^6\frac{ppm_m}{\frac{g}{g}}}
\end{equation*}

\begin{equation*}
\mathrm{[C_6H_6]_{MCL} = 0.005 \,\, ppm_m}
\end{equation*}\\

\item
\begin{equation*}
\mathrm{[C_6H_6]_{MCL} = 0.005 \,\, ppm_m \times \frac{10^3 \, ppb_m}{ppm_m} = 5\, \, ppb_m}
\end{equation*}\\

\item
Molecular weight of benzene (C$_6$H$_6$) = $\mathrm{12\times 6 + 1\times 6 = 78\,\frac{g}{mole}}$
\begin{equation*}
\mathrm{[C_6H_6]_{MCL} = 0.005 \, \frac{mg}{L}\times \frac{1000 \, L}{m^3}\times \frac{1 \, g}{1000 \, mg}\times \frac{1 \, mole}{78 \, g}}
\end{equation*}

\begin{equation*}
\mathrm{[C_6H_6]_{MCL} = 6.4\times 10^{-5} \, \frac{mole}{m^3}}
\end{equation*}


\end{enumerate}

\end{enumerate}

\end{document}