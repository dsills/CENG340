\documentclass[12pt,letterpaper]{article}
%\documentstyle[11pt]{article}
\usepackage[utf8]{inputenc}
\usepackage{amsmath}
\usepackage{xfrac}
\usepackage{amsfonts}
\usepackage{amssymb}
\usepackage[version = 3]{mhchem}
\usepackage{chemstyle}
%%For Table perhaps%%
%\usepackage{graphics}
\usepackage{graphicx}
\usepackage{epstopdf}
%\usepackage{tabularx,ragged2e,booktabs,caption}
%\newcolumntype{C}[1]{>{\Centering}m{#1}}
%\renewcommand\tabularxcolumn[1]{C{#1}}
\usepackage[left=2cm,right=2cm,top=2cm,bottom=2cm]{geometry}
\usepackage{subcaption} 
\usepackage{caption}
\usepackage[colorlinks]{hyperref}
\usepackage[svgnames]{xcolor}
\hypersetup{citecolor=DeepPink4}
\hypersetup{linkcolor=DarkRed}
\hypersetup{urlcolor=DarkBlue}
\usepackage{cleveref}

\begin{document}
\setlength{\parindent}{0cm} 


\frenchspacing

\title {Problem Set 1} 
\author {CENG 340--Introduction to Environmental Engineering\\
Instructor: Deborah Sills}
\date {September 1, 2013}
\maketitle

\section *{Due Date}
Monday 9 September, in class.

\section *{Learning Goals}
\begin{enumerate}
\item Become familiar with the environmental engineering profession.
\item Know where to find information on local water quality.
\item Apply commonly used units to express environmental measurements.
\end{enumerate}


\section *{Questions}
\begin{enumerate}
\item My former colleague, Dr. Gretchen Heavner, works for an environmental engineering firm in Seattle, WA, called \href{http://floydsnider.com}{Floyd\textbar Snider}.  Dr.Heavner will attend our class virtually on September 13th to answer questions about what environmental engineers do.  To prepare, go to Floyd\textbar Snyder's website at \href{http://floydsnider.com}{http://floydsnider.com}, read about their services, expertise, projects, and business approach, and compose two questions for Dr. Heavner (please type your answers).  Make sure your questions are thoughtful, specific, and  demonstrate that you looked carefully at the website.  

\item On Friday, a student in CENG340 asked me whether it's safe to drink the tap water in Lewisburg. I told him that I drink the water, but that it's best to check.   Luckily, the local water utility provides such information on their \href{http://www.amwater.com/ccr/whitedeer.pdf}{website}.  Take a look at the information provided in this document, and answer the following questions:
\begin{itemize}
\item What is the source of the water?
\item Are there any violations?  If not are there any constituents that are at levels close to being violations?
\item If so, are the violations for physical, biological, or chemical constituents?
\end{itemize}


\item Benzene is associated with petroleum products and is typically found in contaminated soil beneath gas stations.  What is the maximum contaminant level (MCL) of benzene (in units of mg/L) allowed in drinking water (hint: read the first two sections of Ch. 10 in the textbook, which were assigned for last Wednesday)?  Express this drinking water standard in terms of (a) ppm$_m$, (b) ppb$_m$, and (c) moles/m$^3$.

%\item (from Mihelcic and Zimmerman and the \textbf{quiz prep assignment}) In 2004, U.S. landfills emitted approximately 6,709 Gg of methane, and wastewater treatment plants emitted 1,758 Gg of methane.  How may Tg  of CO$_2$ equivalents did landfills and wastewater plants emit in 2004.  What percent of the total methane emissions (and greenhouse gas emissions [GHGs]) do these two sources contribute? Total methane emissions in 2004 were 556.7 Tg CO$_2$ equivalents and total GHG emissions were 7074 CO$_2$ equivalents. 

%\item A gasoline station has a patch of contaminated soil over a 45 m$^2$ area and 0.5 m deep. Any samples that exceed Total Petroleum Hydrocarbons (TPH) of 50 mg/kg will require a complete remediation of all soil. If the soil bulk density is 1.5 g/cm$^3$, what is the maximum total mass (in kg) of TPH that may remain on site without cleanup?



\end{enumerate}

\end{document}