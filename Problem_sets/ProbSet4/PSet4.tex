\documentclass[12pt,letterpaper]{article}
%\documentstyle[11pt]{article}
\usepackage[utf8]{inputenc}
\usepackage{amsmath}
\usepackage{xfrac}
\usepackage{amsfonts}
\usepackage{amssymb}
\usepackage[version = 3]{mhchem}
\usepackage{chemstyle}
%%For Table perhaps%%
%\usepackage{graphics}
\usepackage{graphicx}
\usepackage{epstopdf}
\usepackage{tabularx,ragged2e,booktabs,caption}
%\newcolumntype{C}[1]{>{\Centering}m{#1}}
\renewcommand\tabularxcolumn[1]{C{#1}}
\usepackage[left=2cm,right=2cm,top=2cm,bottom=2cm]{geometry}
\usepackage{subcaption} 
\usepackage{caption}
\usepackage[colorlinks]{hyperref}
\usepackage[svgnames]{xcolor}
\hypersetup{citecolor=DeepPink4}
\hypersetup{linkcolor=DarkRed}
\hypersetup{urlcolor=DarkBlue}
\usepackage{cleveref}
\usepackage{enumerate}

\begin{document}
\setlength{\parindent}{0cm} 


\frenchspacing

\title {Problem Set 4} 
\author {CENG 340--Introduction to Environmental Engineering\\
Instructor: Deborah Sills}
\date {September 25, 2013}
\maketitle

\section *{Due Date}
Thursday 3 October, by 5pm.  Bring assignments to my office.  If I'm gone, I'll leave an envelope taped to my door.

\section *{Learning Goals}
\begin{enumerate}
\item Use the law of conservation of mass to write and apply mass balance equations.
\item Apply kinetic equations combined with mass balance equations to analyze natural and engineered systems.
\item Evaluate and compare two reactor types (CMFR and PFR) to treat polluted water.

\end{enumerate}


\section *{Questions}
\begin{enumerate}
\item (11 points) An industrial plant discharges 100 kg/day of liquids into a disposal pond.  Measurements show that 1 kg/day seeps out of the bottom of the pond into the ground and 2 kg/day evaporates into the air.  What is the rate of mass accumulation in the pond?\\
\emph{Answer: 97 kg/day} 

\item 	(11 points) Each day 3780 m$^3$ of wastewater is treated at a municipal wastewater treatment plant.  The influent contains 220 mg/L of suspended solids.  The “clarified” water has a suspended solids concentration of 5 mg/L.  Determine the mass of sludge produced daily from the clarifier. (Sludge = suspended solids removed by clarification from the influent).
\\
\emph{Answer: 813 kg/day}  

\item \textbf{FE Formatted Question}\\
(11 points) The decay of chlorine in a distribution system follows first-order decay with a rate constant of 0.360 d$^{-1}$.  If the concentration of chlorine in a well-mixed storage tank is 1.00 mg/L at time zero, what will the concentration be one day later? Assume no water flows out of the tank.

\begin{enumerate}
\item 0.360 $\mathrm{\frac{mg}{L}}$
\item 0.500 $\mathrm{\frac{mg}{L}}$
\item 0.368 $\mathrm{\frac{mg}{L}}$
\item 0.698 $\mathrm{\frac{mg}{L}}$
\end{enumerate}
Show your work even though you wouldn't have to for the FE.

\item \textbf{FE Formatted Question}\\
(11 points) A 350 $\mathrm{m^3}$ retention pond that holds rainwater from a shopping mall is empty at the beginning of a rainstorm.  The flow rate out of the retention pond must be restricted to 320 L/min to prevent downstream flooding from a 6-hour storm.  What is the maximum flow rate (in L/min) into the pond from a 6-hour storm that will not flood it.

\begin{enumerate}
\item 5,860 $\mathrm{\frac{L}{min}}$
\item 321 $\mathrm{\frac{L}{min}}$
\item 1,290 $\mathrm{\frac{L}{min}}$
\item 7,750 $\mathrm{\frac{L}{min}}$
\end{enumerate}
Show your work even though you wouldn't have to for the FE.


\item \textbf{FE Formatted Question}\\
(11 points) A pipeline carrying 0.50 MGD of a 35,000 mg/L brine solution (NaCl) across a creek ruptures.  The flow rate of the creek is 2.80 MGD.  If the salt concentration in the creek is 175 mg/L, what is the concentration of salt in the creek after the pipeline discharge mixes completely with the creek water?

\begin{enumerate}
\item $\mathrm{1.80\times 10^4\, \frac{mg}{L}}$
\item $\mathrm{1.75\times 10^2\, \frac{mg}{L}}$
\item $\mathrm{5.45\times 10^3\, \frac{mg}{L}}$
\item $\mathrm{6.43\times 10^3\, \frac{mg}{L}}$
\end{enumerate}
Show your work even though you wouldn't have to for the FE.


\item (15 points) A freshwater pond has a well-mixed volume equal to 106 $\mathrm{m^3}$.   The pond is fed by a single stream and drains by another stream.  There is negligible input/output other than these two streams.  Flow in and out is 103 $\mathrm{\frac{m^3}{day}}$.  The inflow stream contains a contaminant with concentration equal to 3.4 mg/L. 

\begin{enumerate}
\item Determine the steady-state concentration of the contaminant in the pond if the contaminant decays in the pond at a rate equal to 0.001 $\mathrm{\frac{mg}{L\times day}}$.\\
\emph{Answer: 2.4 mg/L}
 
\item Determine the steady-state concentration of the contaminant in the pond if the contaminant decays in the pond at a first order rate given by:
\begin{equation*}
\mathrm{r = k\times C}
\end{equation*} 

where k = 0.01 $\mathrm{\frac{1}{day}}$ and C has units of $\mathrm{\frac{mg}{L}}$.\\
\emph{Answer: 0.309 mg/L}
\end{enumerate}

\item From \emph{Introduction to Environmental Engineering} by Davis and Cornwell.\\
(15 points) A sewage lagoon that has a surface area of 10 ha and a depth of 1 m is receiving 8,640 $\mathrm{\frac{m^3}{d}}$ of sewage containing 100 $\mathrm{\frac{mg}{L}}$ of a biodegradable contaminant.  
\begin{enumerate}
\item Assuming the lagoon is well mixed and there are no losses or gains of water in the lagoon except for the sewage influent and effluent, what biodegradation reaction rate coefficient, k (d$^{-1}$), must be achieved for a first-order reaction?\\
\emph{Answer: k = 0.35 d$^{-1}$}

\item Solve the same problem but assume that instead of one lagoon, there are two well-mixed lagoons in series.  Each lagoon has a surface area of 5 ha and a depth of 1 m.\\
\emph{Answer: k = 0.21 d$^{-1}$}

\item Assume that the process that produces sewage suddenly stops and clean water begins to flow into the single pond (from Part a).  Use Excel (or any software you like) and plot the effluent concentration as a function of time at 1 day intervals for 10 days. 
 
\end{enumerate}
\item From Nazaroff and Alvarez Cohen.\\
(15 points) Design a steady-state reactor for the treatment of 100 mg/L toluene in water.  Assume that toluene is degraded by a first-order reaction with a rate coefficient of 0.8 $\mathrm{day^{-1}}$.
\begin{enumerate}
\item Calculate the hydraulic retention time (HRT = $\mathrm{\theta = \frac{volume}{flow\, rate}}$ required to achieve 96 percent removal in a CMFR. \emph{Answer $\theta$ = 30 days}
\item Calculate the HRT required to achieve 96 percent removal in a PFR. \emph{Answer $\theta$ = 4 days}
\item (Extra Credit) Compare the required HRTs for each reactor and briefly justify why they are different.
\end{enumerate}
\end{enumerate}
\end{document}