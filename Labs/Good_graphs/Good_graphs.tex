\documentclass[12pt,letterpaper]{article}
\usepackage[utf8]{inputenc}
\usepackage{amsmath}
\usepackage{amsfonts}
\usepackage{amssymb}
\usepackage[version = 3]{mhchem}
\usepackage{chemstyle}
%%For Table perhaps%%
%\usepackage{graphics}
\usepackage{graphicx}
\usepackage{epstopdf}
%\usepackage{tabularx,ragged2e,booktabs,caption}
%\newcolumntype{C}[1]{>{\Centering}m{#1}}
%\renewcommand\tabularxcolumn[1]{C{#1}}
\usepackage[left=2cm,right=2cm,top=2cm,bottom=2cm]{geometry}


\begin{document}
\setlength{\parindent}{0cm} 


\frenchspacing

\title {\textbf{What makes a good graph?}} 

\author {Deborah Sills}
%\date {9 February 2013}
\maketitle
Modified from http://www.uwex.edu/ces/csreesvolmon/pdf/trainings/08CSREESDataPresent.pdf
\begin{enumerate}

\item Simple, clear, and appropriate axis labels that include units
\item Elements that allow the reader to get the point
\item A legend explaining graph elements
\item A scale appropriate to the data
\item Reveals a story
\item Minimum of clutter
\end{enumerate}



\end{document}