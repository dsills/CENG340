\documentclass[12pt,letterpaper]{article}
\usepackage[utf8]{inputenc}
\usepackage{amsmath}
\usepackage{amsfonts}
\usepackage{amssymb}
\usepackage[version = 3]{mhchem}
\usepackage{chemstyle}
%%For Table perhaps%%
%\usepackage{graphics}
\usepackage{graphicx}
\usepackage{epstopdf}
%\usepackage{tabularx,ragged2e,booktabs,caption}
%\newcolumntype{C}[1]{>{\Centering}m{#1}}
%\renewcommand\tabularxcolumn[1]{C{#1}}
\usepackage[left=2cm,right=2cm,top=2cm,bottom=2cm]{geometry}


\begin{document}
\setlength{\parindent}{0cm} 


\frenchspacing

\title {\textbf{Lab 6 -- Nonlinear Curve Fitting-Part II} \\ \vspace{2 mm} {\large \textbf{CO$_2$ Delivery for the Cultivation of Algae}}}

\author {CENG340---Introduction to Environmental Engineering}
\date {October 8, 2013}
\maketitle

\section *{Due Date}
\textbf{Submit your memo via email before lab on October 22.}
\section *{Learning Objectives}
\begin{enumerate}

\item Communicate the results of a combined laboratory and modeling study in a formal memo to a client.\ 
\item Create, discuss, and analyze high-quality figures that that contain graphs with data and model fits.\
\item Learn to fit laboratory data to mathematical models using non-linear curve fitting.\


\end{enumerate}

\section *{Overview}
This lab is an continuation of Lab 3, where you learned to fit non-linear data to mathematical models, and present data and models fit in high-quality plots.  In this assignment you will extend what you learned in Lab 3 and incorporate a plot into a clear and concise memo to a client. 

\section *{Rationale}
\begin{enumerate}
\item To be successful, engineers need to communicate results of preliminary experiments and final designs clearly and concisely to their colleagues, supervisors, and clients.

\item Figures or graphs can be an effective way to display data, but only if they're done well. A good figure should allow the reader to easily discern the point you are trying to make.  For more suggestion on creative high-quality figures, please refer to the DO's and DONT's document prepared by Prof. Malusis.

\item Environmental engineers use mathematical models that describe physical, chemical, and biological phenomena to analyze environmental systems and design treatment technologies.  Such models can be used to predict the behavior of a treatment system, or, in some cases, may be used to enhance the visual display of data.

\end{enumerate}
 

\section *{Assignment---CO$_2$ Delivery}

Bioprocess Algae LLC is designing a CO$_2$ delivery system for their raceway ponds, which are used to grow microalgae that will be converted into jet fuel. Since the cost of delivering CO$_2$ to raceway ponds represents about one-third of the total cost of growing algae for fuel production \cite{Lund2010}, a well designed CO$_2$ delivery system is critical for producing biofuel in an economically viable manner.\\  

Bioprocess Algae contracted the engineering firm that you work for to conduct a laboratory study that will determine the effect of the three following designs  on the rate of CO$_2$ input into raceway ponds: (1) mixing with a paddlewheel, (2) CO$_2$ sparging with a fine (small) diameter diffuser, and (3) CO$_2$ sparging with a coarse (large) diameter diffuser.

\subsection *{Assignment}

The data file titled ``CO$_2$ Delivery'' is located at blahblahblah drive blah %/ facultystaff on ‘netspace’ (T:)/dls054/public/CENG340/Week5_LabFiles/.
The file contains three sets of CO$_2$ concentration versus time data from an open raceway pond, where  CO$_2$ was delivered via the three methods described above.

\begin{enumerate}
\item mixing only, 
\item sparging with coarse (large) bubbles, and 
\item sparging with fine (small) bubbles.
\end{enumerate}

\begin{itemize} 
\item Use KaleidaGraph (or another software of your choice) to fit the three-parameter (C$_{sat}$, C$_0$, and k) non-linear mass transfer equation (Eq.1) that we used in lab last week, to each data set.\\

\begin{equation}
C = C_{sat} - (C_{sat} - C_0)e^{-kt}
\end{equation}\\

where C = aqueous concentration of dissolved CO$_2$ at time = t,\\

C$_{sat}$ = aqueous concentration of CO$_2$ when the water in the pond is in equilibrium with the atmosphere,\\

C$_0$ = aqueous concentration of CO$_2$ at time = 0, and\\

k = first order rate coefficient.\\



\item Create one plot that shows all three sets of C vs. t data points, and a line/curve that illustrates the model fit to each data set. 
\item Think about how well the model fits the data and if you think the model is appropriate or not. Note that we will not conduct proper statistical tests for goodness of fit, but you should be able to visually assess whether the model is appropriate or not, and you should include this assessment in your memo.
\item Create a column graph with three columns, showing the value of the mass transfer coefficient (k) on the y-axis that corresponds to each aeration method on the x-axis.
 
\end{itemize}


\subsection *{Deliverables}


Summarize your work in a memo addressed to Dr. Toby Ahrens, the Chief Science Officer of  Bioprocess Algae. For this assignment include the following sections: (1) Introduction, (2) Methods,(3) Results and Discussion (this is where you'll include the figures you created), and (4) Conclusions.\\ 

In this assignment make sure you think about your audience (a client), what he is hoping to receive from you (a design recommendation?), and how you will convince him that your design choice is appropriate. Refer to the instructions handed out at the beginning of the semester for  formatting and writing style, and the sample memo.  In addition refer to the comments you received on Lab Report 3 and please contact me if you have questions.  Pay attention to the conclusions you draw and make sure that they come out of the data and model fits you present in the figures.\\



\begin{thebibliography}{9}


\bibitem{Lund2010}
Lundquist, Tryg J and Woertz, Ian C and Quinn, NWT and Benemann, John R
 ``A realistic technology and engineering assessment of algae biofuel production,''
  \emph{Energy Biosciences Institute}, 2010.


\end{thebibliography}


\end{document}