\documentclass[11pt,letterpaper]{article}
%\documentstyle[11pt]{article}
\usepackage[utf8]{inputenc}
\usepackage{amsmath}
\usepackage{xfrac}
\usepackage{amsfonts}
\usepackage{amssymb}
\usepackage[version = 3]{mhchem}
\usepackage{chemstyle}
\usepackage{graphicx}
\usepackage{epstopdf}
\usepackage{tabularx,ragged2e,booktabs,caption}
%\newcolumntype{C}[1]{>{\Centering}m{#1}}
%\renewcommand\tabularxcolumn[1]{C{#1}}
%\usepackage[left=2cm,right=2cm,top=2cm,bottom=2cm]{geometry}
\usepackage{subcaption} 
\usepackage{caption}
\usepackage[left=2cm,right=2cm,top=1cm,bottom=2cm]{geometry}
%\usepackage{siunitx}



\begin{document}
\setlength{\parindent}{0cm} 



\frenchspacing

% Default margins are too wide all the way around. I reset them here
%\setlength{\topmargin}{-.5in}
%\setlength{\textheight}{9in}
%\setlength{\oddsidemargin}{.125in}
%\setlength{\evensidemargin}{.125in}
\setlength{\textwidth}{6.25in}

\title {\Large{\textbf{Water Treatment Unit Operations---Design Paramters}}\\ \large{To be used as part of PSet6\\CENG 340--Introduction to Environmental Engineering\\
Instructor: Deborah Sills\\ \textbf{In Class: 21 October, 2013}}}

\author {}
\date {}
\maketitle

\vspace{-2cm}
\subsection *{Introduction}
You have been asked to design a water treatment facility to meet the following criteria:

\begin{itemize}
\item Design capacity = 3.25 MGD
\item Source is river water with an initial turbidity of 10 NTU, an alkalinity concentration of 50 mg/L, at 10 $^0$C, with dynamic viscosity of $\mathrm{1.307 \times 10^{-3}\, \frac{N\times s}{m^2}}$ and pH = 7.
\item Unit operations: coagulation (rapid mix), flocculation, sedimentation, rapid sand filtration, disinfection
\item Additional constraints: units must be sized according to acceptable ranges.  Design must accommodate maintenance and repair.\\
\end{itemize}

\vspace{-0.2in}

\subsection *{Approach}
\begin{enumerate}
\item Identify contaminants to be removed
\item Prepare a block-flow diagram of necessary unit operations, in proper order
\item Complete a conceptual design of each unit operation
\end{enumerate}

\vspace{0.2in}


\begin{minipage}{\linewidth}
\centering
\captionof{table}{\textbf{Typical values used in design of water treatment systems} (adapted from our textbook).} \label{tab:title}

\begin{tabular}{|p{50mm}|p{70mm}|p{30mm}|}\toprule[1.25pt]
\bf Unit Operation	& \bf Design Basis 	& \bf Calculate	\\\midrule
Coagulation--rapid-mix tank	& $\mathrm{\theta =}$ 1--2 min \newline $\bar{G}$ = 600--1000 s$^{-1}$ \newline Coagulant type & Volume \newline Number of tanks \newline Mixing Power (P) \newline Coagulant dose \newline Alkalinity req'd\\ \hline\

Flocculation Tank 	& $\mathrm{\theta =}$ 10--30 min \newline $\bar{G}$ = 20--50 s$^{-1}$ (horiz. paddle) \newline $\bar{G}$ = 10--80 s$^{-1}$ (vertical shaft)	& Volume \newline Number of tanks \newline Mixing power (P)\\ \hline\

Sedimentation tanks & $\mathrm{\theta =}$ 2--4 h \newline OFR = 700--1400 gpd/ft$^2$ \newline Weir loading rate = 20,000 gpd/ft & Area \newline Volume \newline Number of tanks \newline Weir length\\ \hline\

Filtration (rapid sand) & Hyd. loading rate = 2--6 gpm/ft$^2$ \newline Depth = 2--6 ft & Area \newline Volume \newline Number of filters\\ \hline\

Chlorination & $\mathrm{\theta_{min} =}$ 15 min (at peak hourly flow) \newline $\mathrm{\theta_{min} =}$ 30 min (at average hourly flow) & Volume \newline Chlorine dose\\ 


\bottomrule[1.25pt]

\end {tabular}\par
\end{minipage}\\

\end{document}