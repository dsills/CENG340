\documentclass[11pt,letterpaper]{article}
%\documentstyle[11pt]{article}
\usepackage[utf8]{inputenc}
\usepackage{amsmath}
\usepackage{xfrac}
\usepackage{amsfonts}
\usepackage{amssymb}
\usepackage[version = 3]{mhchem}
\usepackage{chemstyle}
\usepackage{graphicx}
\usepackage{epstopdf}
%\usepackage{tabularx,ragged2e,booktabs,caption}
%\newcolumntype{C}[1]{>{\Centering}m{#1}}
%\renewcommand\tabularxcolumn[1]{C{#1}}
%\usepackage[left=2cm,right=2cm,top=2cm,bottom=2cm]{geometry}
\usepackage{subcaption} 
\usepackage{caption}
\usepackage[left=2cm,right=2cm,top=1cm,bottom=2cm]{geometry}
%\usepackage{siunitx}



\begin{document}
\setlength{\parindent}{0cm} 



\frenchspacing

% Default margins are too wide all the way around. I reset them here
%\setlength{\topmargin}{-.5in}
%\setlength{\textheight}{9in}
%\setlength{\oddsidemargin}{.125in}
%\setlength{\evensidemargin}{.125in}
\setlength{\textwidth}{6.25in}

\title {\Large{\textbf{Mass Balance--CMFR with Conservative Pollutant}}\\ \large{CENG 340--Introduction to Environmental Engineering\\
Instructor: Deborah Sills\\ \textbf{In Class: September 23, 2013}}}

\author {}
\date {}
\maketitle

\vspace{-1.5cm}


\textbf{( Example 4.5 in \emph{Environmental Engineering} by Mihelcic and Zimmerman)}\\

A completely mixed stirred reactor (CMFR) contains clean water prior to being started.  After start-up, a waste stream containing 100 mg/L of a conservative pollutant is added to the reactor at a flow rate of 50 $\mathrm{\frac{m^3}{day}}$.  The volume of the reactor is 500 m$^3$. What is the concentration exiting the reactor as function of time after it is started.


\section *{Step 1:} 
Determine whether the system is at steady state or not.

\vspace{0.2in}

\section *{Step 2:} 
Draw a mass balance diagram.

\vspace{1.5in}

\section *{Step 3:}
Write a mass balance equation:

\vspace{0.8in}
\section *{Step 4:} 
Solve the mass balance equation:

\end{document}