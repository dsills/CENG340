\documentclass[12pt,letterpaper]{article}
%\documentstyle[11pt]{article}
\usepackage[utf8]{inputenc}
\usepackage{amsmath}
\newcommand{\var}[1]{{\operatorname{#1}}}
%\usepackage{breqn}
\usepackage{xfrac}
\usepackage{amsfonts}
\usepackage{amssymb}
\usepackage[version = 3]{mhchem}
\usepackage{chemstyle}
%%For Table perhaps%%
%\usepackage{graphics}
\usepackage{graphicx}
\usepackage{epstopdf}
%\usepackage{tabularx,ragged2e,booktabs,caption}
%\newcolumntype{C}[1]{>{\Centering}m{#1}}
%\renewcommand\tabularxcolumn[1]{C{#1}}
\usepackage[left=2cm,right=2cm,top=0.5cm,bottom=2cm]{geometry}
\usepackage{subcaption} 
\usepackage{caption}
%\usepackage{siunitx}
%\usepackage{subfig}



\begin{document}
\setlength{\parindent}{0cm} 


\frenchspacing


% Default margins are too wide all the way around. I reset them here
%\setlength{\topmargin}{-.5in}
%\setlength{\textheight}{9in}
%\setlength{\oddsidemargin}{.125in}
%\setlength{\textwidth}{6.25in}




\title {\large{In Class Problem} \\ 
{\large \textbf{Kinetics}\\CENG 340--Introduction to Environmental Engineering\\
Instructor: Deborah Sills\\September 16, 2013}}

\author{}

\date {}
\maketitle

\vspace{-0.8in}



\emph {Modified from Mihelcic and Zimmerman}\\

Nitrogen dioxide (NO$_2$), an air pollutant produced when N$_2$ in air reacts with O$_2$ during fuel combustion, can be destroyed via photochemical reactions, which result in the formation of ozone.  In one study, NO$_2$  concentrations decreased from 5 ppm$_v$ to 2 ppm$_v$ in four minutes due to exposure to light.  

\begin{enumerate}
\item What is the first-order rate constant for this reaction?
\vspace{3in}
\item What was the half-life of NO$_2$ during this study?
\end{enumerate}


\end{document}