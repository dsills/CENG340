\documentclass[12pt,letterpaper]{article}
%\documentstyle[11pt]{article}
\usepackage[utf8]{inputenc}
\usepackage{amsmath}
\usepackage{xfrac}
\usepackage{amsfonts}
\usepackage{amssymb}
\usepackage[version = 3]{mhchem}
\usepackage{chemstyle}
%%For Table perhaps%%
%\usepackage{graphics}
\usepackage{graphicx}
\usepackage{epstopdf}
%\usepackage{tabularx,ragged2e,booktabs,caption}
%\newcolumntype{C}[1]{>{\Centering}m{#1}}
%\renewcommand\tabularxcolumn[1]{C{#1}}
\usepackage[left=2cm,right=2cm,top=2cm,bottom=2cm]{geometry}
\usepackage{subcaption} 
\usepackage{caption}
\usepackage[colorlinks]{hyperref}
\usepackage[svgnames]{xcolor}
\hypersetup{citecolor=DeepPink4}
\hypersetup{linkcolor=DarkRed}
\hypersetup{urlcolor=DarkBlue}
\usepackage{cleveref}

\begin{document}
\setlength{\parindent}{0cm} 


\frenchspacing






\title {Prep for Quiz 1---Environmental Measurements} 
\author {CENG 340--Introduction to Environmental Engineering\\
Instructor: Deborah Sills}
%\date {, 2013}
\maketitle

\section *{Rationale}
A lecture on the units used to express concentrations of chemicals and biological entities (e.g., coliforms) would bore you and me to death. Therefore, I'm asking you to engage with the textbook and answer the following questions in preparation for an in-class quiz on Monday 9/2. Note that I will not collect this preparatory assignment, but if you work through these problems, you should be able to succeed on the exam.

\section *{Learning Goals}
\begin{enumerate}
\item 
\end{enumerate}

\section *{Questions}
A subset of similar questions will appear on Wednesday's quiz.


\begin{enumerate}
\item Vinyl chloride is classified as a known carcinogen by the U.S. Environmental Protection Agency (EPA), and according to \href{http://water.epa.gov/drink/contaminants/basicinformation/vinyl-chloride.cfm}{their website}, "EPA has set an enforceable regulation for vinyl chloride, called a maximum contaminant level (MCL), at 0.002 mg/L or 2 ppb."  Prove that 0.002 mg/L equals 2 ppb. 

\item (modified from Mihelcic and Zimmerman) The EPA regulates nitrate in drinking water with a MCL of 10 mg/L-N.  Babies under the age of 6 months who drink water with nitrate concentrations higher than the MCL could be come very ill and evn die from a syndrome called "blue-baby syndrome."  If a water sample contains 10 mg NO$_3^{-2}$/L, does it violate the MCL?  Also convert the MCL to units of (a)ppm$_m$, (b) moles/L, and (c) ppb$_m$. 




\item \textbf{Who loves hockey?} (modified from Mihelcic and Zimmerman)\\
Ice resurfacing machines (aka Zambonis) use internal combustion vehicles that give off exhaust containing carbon monoxide (CO) and nitrogen oxides (NO$_x$).  The outdoor air quality 1-h standard of CO is set at 35 mg/m$^3$. Average CO concentrations measured at Lynah Rink (at Cornell University) have been reported to be as high as 115 ppm$_v$ and as low as 35 ppm$_v$.  Should Prof. Sills be concerned about spending 1 h at Lynah Rink, watching Cornell Women's Ice Hockey Team play (and hopefully beat) Northeastern on October 19th?  Assume the temperature equals 20$^0$C. 

\item What is the molar concentration of 10 grams/liter for each of the following chemicals?
\begin{itemize}
\item NaOH     
\item Na$_2$SO$_4$   
\item K$_2$Cr$_2$O$_7$   
\item KCl  
\end{itemize}

\item A lake water sample has the following cation composition:
\begin{itemize}
\item Ca$^{2+}$
\item Na$^{+}$
\item Mg$^{2+}$
\item Mn$^{2+}$
\end{itemize}
Determine the hardness of the water (express the hardness as mg/L of CaCO3).
 
\item (modified from Mihelcic and Zimmerman) Coliform bacteria (for example \emph{E. coli}) are excreted in large numbers in human and animal feces.  Water that meets a standard of less than one coliform per 100 mL is considered safe for human consumption.  Is a 1 m$^3$ water sample that contains 9000 coliforms safe for human consumption?  \textbf{Show your work.}

\item A gasoline station has a patch of contaminated soil over a 45 m$^2$ area and 0.5 m deep. Any samples that exceed Total Petroleum Hydrocarbons (TPH) of 50 mg/kg will require a complete remediation of all soil. If the soil bulk density is 1.5 g/cm$^3$, what is the maximum total mass (in kg) of TPH that may remain on site without cleanup?


\end{enumerate}

\end{document}