\documentclass[12pt,letterpaper]{article}
%\documentstyle[11pt]{article}
\usepackage[utf8]{inputenc}
\usepackage{amsmath}
\usepackage{xfrac}
\usepackage{amsfonts}
\usepackage{amssymb}
\usepackage[version = 3]{mhchem}
\usepackage{chemstyle}
%%For Table perhaps%%
%\usepackage{graphics}
\usepackage{graphicx}
\usepackage{epstopdf}
%\usepackage{tabularx,ragged2e,booktabs,caption}
%\newcolumntype{C}[1]{>{\Centering}m{#1}}
%\renewcommand\tabularxcolumn[1]{C{#1}}
\usepackage[left=2cm,right=2cm,top=0.5cm,bottom=2cm]{geometry}
\usepackage{subcaption} 
\usepackage{caption}
\usepackage[colorlinks]{hyperref}
\usepackage[svgnames]{xcolor}
\hypersetup{citecolor=DeepPink4}
\hypersetup{linkcolor=DarkRed}
\hypersetup{urlcolor=DarkBlue}
\usepackage{cleveref}

\begin{document}
\setlength{\parindent}{0cm} 


\frenchspacing


\title {\Large In Class Post Quiz 1 Pre-Quiz 2---Environmental Measurements} 
\author {CENG 340--Introduction to Environmental Engineering\\
Instructor: Deborah Sills}
\date {September 6, 2013}
\maketitle


\section *{What are PVC pipes made from?}

\subsection *{Vinyl Chloride, C$_2$H$_3$Cl}

The Environmental Protection Agency regulates vinyl chloride (VC) under the Clean Air Act and the Safe Drinking Water Act.  VC has an MCL of 0.002 ppm$_m$ and a 3-h outdoor air quality standard of 10 ppm$_v$.\\

A manager of a facility that produces polyvinyl chloride (PVC) informs you that a malfunctioning valve caused 60 g of VC to be discharged into an indoor swimming pool. \textbf{She wants to know if the resulting concentrations of VC in the water and air violate the MCL and outdoor air quality standard, respectively.}\\  

Since VC is very volatile, assume that 58 g of the VC volatilzed out of the water into the air surrounding the indoor pool, and 2 g remained dissolved in water.  (Note that next week you will learn to calculate how volatile compounds like VC partition between air and water.)\\

Additional useful information:

\begin{enumerate}

\item The pool has a volume of 100 m$^3$ and the indoor pool area has an air volume of 1000 m$^3$. 

\item Temperature and pressure equal 25 $^0$C and 1 atm, respectively.

\item Temperature in Kelvin (K) = temperature in degrees Celsius ($^0$C) + 273.15; 

\item MW$\mathrm{_C}$ = 12 g/mole; MW$\mathrm{_H}$ = 1 g/mole; MW$\mathrm{_{Cl}}$ = 35.5 g/mole

\item The ideal gas constant R = $8.205\times 10^{-5}\, \mathrm{\frac{m^3 \times atm}{mole \times Kelvin}}$. 


\end{enumerate}



\end{document}