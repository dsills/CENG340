\documentclass[12pt,letterpaper]{article}
%\documentstyle[11pt]{article}
\usepackage[utf8]{inputenc}
\usepackage{amsmath}
\usepackage{xfrac}
\usepackage{amsfonts}
\usepackage{amssymb}
\usepackage[version = 3]{mhchem}
\usepackage{chemstyle}
%%For Table perhaps%%
%\usepackage{graphics}
\usepackage{graphicx}
\usepackage{epstopdf}
\usepackage{tabularx,ragged2e,booktabs,caption}
%\newcolumntype{C}[1]{>{\Centering}m{#1}}
\renewcommand\tabularxcolumn[1]{C{#1}}
\usepackage[left=2cm,right=2cm,top=0.5cm,bottom=2cm]{geometry}
\usepackage{subcaption} 
\usepackage{caption}
\usepackage[colorlinks]{hyperref}
\usepackage[svgnames]{xcolor}
\hypersetup{citecolor=DeepPink4}
\hypersetup{linkcolor=DarkRed}
\hypersetup{urlcolor=DarkBlue}
\usepackage{cleveref}
\usepackage{enumerate}

\begin{document}
\setlength{\parindent}{0cm} 


\frenchspacing

\title {In Class Problem---Fish Kill \\ {\Large \textbf{Acid--Base Chemistry}}}
\author {CENG 340--Introduction to Environmental Engineering\\
Instructor: Deborah Sills\\September 11, 2013}
\date {}
\maketitle


\textbf{\emph{Concentrations of NH$_3$ (a weak acid) that are higher than 0.2 mg/L are toxic to fish.}}\\ 

Effluent from a wastewater treatment plant (WWTP)---with pH = 7 and a total ammonia concentration of 5 \textbf{mg-N}/L---is discharged to a stream that is popular with fisherman.  Ammonia consists of two species: NH$_4^+$ (a weak acid) and  NH$_3$ (its conjugate base). The concentrations of NH$_4^+$ and NH$_3$ are related to each other through the following equilibrium reaction:

\begin{align*}
\cee{[NH_4^+] &<=>[K_a] [NH_3] + [H^+]}
\end{align*}

where K$_a$ = $10^{-9.3}$, or pK$_a$ = 9.3.

\begin{enumerate}

\item  You've been asked to use the total ammonia data reported above to calculate the concentration of NH$_3$ in the WWTP effluent to ensure that [NH$_3$] is lower than 0.2 mg/L.



\end{enumerate}

\end{document}