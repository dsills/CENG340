\documentclass[11pt,letterpaper]{article}
%\documentstyle[11pt]{article}
\usepackage[utf8]{inputenc}
\usepackage{amsmath}
\usepackage{xfrac}
\usepackage{amsfonts}
\usepackage{amssymb}
\usepackage[version = 3]{mhchem}
\usepackage{chemstyle}
\usepackage{graphicx}
\usepackage{epstopdf}
%\usepackage{tabularx,ragged2e,booktabs,caption}
%\newcolumntype{C}[1]{>{\Centering}m{#1}}
%\renewcommand\tabularxcolumn[1]{C{#1}}
%\usepackage[left=2cm,right=2cm,top=2cm,bottom=2cm]{geometry}
\usepackage{subcaption} 
\usepackage{caption}
\usepackage[left=2cm,right=2cm,top=1cm,bottom=2cm]{geometry}
%\usepackage{siunitx}



\begin{document}
\setlength{\parindent}{0cm} 



\frenchspacing

% Default margins are too wide all the way around. I reset them here
%\setlength{\topmargin}{-.5in}
%\setlength{\textheight}{9in}
%\setlength{\oddsidemargin}{.125in}
%\setlength{\evensidemargin}{.125in}
\setlength{\textwidth}{6.25in}

\title {\Large{\textbf{Big Biogas Boom---How much alkalinity?}}\\ \large{CENG 340--Introduction to Environmental Engineering\\
Instructor: Deborah Sills\\ \textbf{In Class: September 13/16, 2013}}}

\author {}
\date {}
\maketitle

\vspace{-1.5cm}

\large{Name:}

\vspace{-0.3cm}

Anaerobic digesters that convert organic waste (e.g., animal manure, food waste, and sewage sludge) often fail because of insufficient alkalinity required to buffer the digester and maintain a pH of 7 or above.  At pH < 7, methanogens, the organisms that produce methane in digesters, are not active. Alkalinity as bicarbonate is often added to digesters to prevent the pH from dropping below 7. 

You are designing an anaerobic digester that will convert food waste, composed primarily of carbohydrate, to biogas.  Note that carbohydrate can be approximated as glucose and an unbalanced reaction for glucose conversion to methane is presented in Eq.(1):


Calculate the concentration of bicarbonate alkalinity in mg/L as CaCO$_3$ that is needed in the digester to maintain a pH of 7.   








\end{document}