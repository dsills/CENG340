\documentclass[11pt,letterpaper]{article}
%\documentstyle[11pt]{article}
\usepackage[utf8]{inputenc}
\usepackage{amsmath}
\usepackage{xfrac}
\usepackage{amsfonts}
\usepackage{amssymb}
\usepackage[version = 3]{mhchem}
\usepackage{chemstyle}
\usepackage{graphicx}
\usepackage{epstopdf}
%\usepackage{tabularx,ragged2e,booktabs,caption}
%\newcolumntype{C}[1]{>{\Centering}m{#1}}
%\renewcommand\tabularxcolumn[1]{C{#1}}
%\usepackage[left=2cm,right=2cm,top=2cm,bottom=2cm]{geometry}
\usepackage{subcaption} 
\usepackage{caption}
\usepackage[left=2cm,right=2cm,top=1cm,bottom=2cm]{geometry}
%\usepackage{siunitx}



\begin{document}
\setlength{\parindent}{0cm} 



\frenchspacing

% Default margins are too wide all the way around. I reset them here
%\setlength{\topmargin}{-.5in}
%\setlength{\textheight}{9in}
%\setlength{\oddsidemargin}{.125in}
%\setlength{\evensidemargin}{.125in}
\setlength{\textwidth}{6.25in}

\title {\Large{\textbf{Mass Balance--Salt in a Storm Sewer}}\\ \large{CENG 340--Introduction to Environmental Engineering\\
Instructor: Deborah Sills\\ \textbf{In Class: September 20, 2013}}}

\author {}
\date {}
\maketitle

\vspace{-1.5cm}


\textbf{(adapted from \emph{Environmental Engineering} by Davis and Cornwell)}\\
A storm sewer is carrying snow melt containing 1,200 g/L of sodium chloride into a small stream.  The stream has a naturally occurring sodium chloride concentration of 20 mg/L.  If the storm sewer flow rate is 2000 L/min and the stream flow rate is 2.0 m$^3$/s, what is the concentration of the salt in the stream after the discharge point?  Assume that the sewer flow and stream flow are completely mixed, and that the salt is a conservative substance (it does not react) and that the system is at steady state?

\section *{Step 1:} 
Underline any words or phrases that are unclear to you--i.e., even if you understand the English, there are a few key phrases in the problem that we have not covered.

\vspace{1in}

\section *{Step 2:} 
Draw a mass balance diagram.

\vspace{2in}

\section *{Step 3:}
Write a mass balance equation:

\pagebreak
\section *{Step 4:} 
Solve the mass balance equation:

\end{document}