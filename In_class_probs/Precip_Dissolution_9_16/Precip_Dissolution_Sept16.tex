\documentclass[12pt,letterpaper]{article}
%\documentstyle[11pt]{article}
\usepackage[utf8]{inputenc}
\usepackage{amsmath}
\newcommand{\var}[1]{{\operatorname{#1}}}
%\usepackage{breqn}
\usepackage{xfrac}
\usepackage{amsfonts}
\usepackage{amssymb}
\usepackage[version = 3]{mhchem}
\usepackage{chemstyle}
%%For Table perhaps%%
%\usepackage{graphics}
\usepackage{graphicx}
\usepackage{epstopdf}
%\usepackage{tabularx,ragged2e,booktabs,caption}
%\newcolumntype{C}[1]{>{\Centering}m{#1}}
%\renewcommand\tabularxcolumn[1]{C{#1}}
\usepackage[left=2cm,right=2cm,top=0.5cm,bottom=2cm]{geometry}
\usepackage{subcaption} 
\usepackage{caption}
%\usepackage{siunitx}
%\usepackage{subfig}



\begin{document}
\setlength{\parindent}{0cm} 


\frenchspacing


% Default margins are too wide all the way around. I reset them here
%\setlength{\topmargin}{-.5in}
%\setlength{\textheight}{9in}
%\setlength{\oddsidemargin}{.125in}
%\setlength{\textwidth}{6.25in}




\title {\large{In Class Problems} \\ 
{\large \textbf{Precipitation--Dissolution Equilibrium \& Kinetics}\\CENG 340--Introduction to Environmental Engineering\\
Instructor: Deborah Sills\\September 16, 2013}}

\author{}

\date {}
\maketitle

\vspace{-1.0in}

\begin{enumerate}

\item \emph {Modified from Mihelcic and Zimmerman}\\
Ingesting cadmium may lead to kidney damage.  According to the EPA, major sources of cadmium in drinking water include corrosion of galvanized pipes; erosion of natural deposits; discharge from metal refineries; and runoff from waste batteries and paints.\\

One method to remove metals, such as cadmium, from water is to raise the pH and cause them to precipitate as their metal hydroxides.  The precipitation--dissolution equilibrium relationship for cadmium hydroxide is described by the following equation: 

\begin{align*}
\cee{[Cd(OH)_2]  &<=>[K_{sp}] [Cd^{2+}] + 2[OH^-]}
\end{align*}

where pK$_{sp}$ = 13.85\\

\begin{enumerate}
\item In an attempt to remove cadmium from water by precipitating cadmium hydroxide, the pH of water was raised from 6.8 to 8.0.  Was the dissolved cadmium concentration reduced to below 100 mg/L at the final pH? 

\vspace{2.5in}

\item What pH is required to reduce cadmium concentration to below the MCL of 5 ppb?\\\\\\\\

\vspace{2in}

\end{enumerate}

\pagebreak

\vspace{5in}

\item \emph {Modified from Mihelcic and Zimmerman}\\

Nitrogen dioxide (NO$_2$), an air pollutant produced when N$_2$ in air reacts with O$_2$ during fuel combustion, can be destroyed via photochemical reactions, which result in the formation of ozone.  In one study, NO$_2$  concentrations decreased from 5 ppm$_v$ to 2 ppm$_v$ in four minutes due to exposure to light.  

\begin{enumerate}
\item What is the first-order rate constant for this reaction?
\vspace{3in}
\item What was the half-life of NO$_2$ during this study?
\end{enumerate}

\end{enumerate}
\end{document}