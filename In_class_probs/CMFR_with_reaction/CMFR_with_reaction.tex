\documentclass[11pt,letterpaper]{article}
%\documentstyle[11pt]{article}
\usepackage[utf8]{inputenc}
\usepackage{amsmath}
\usepackage{xfrac}
\usepackage{amsfonts}
\usepackage{amssymb}
\usepackage[version = 3]{mhchem}
\usepackage{chemstyle}
\usepackage{graphicx}
\usepackage{epstopdf}
%\usepackage{tabularx,ragged2e,booktabs,caption}
%\newcolumntype{C}[1]{>{\Centering}m{#1}}
%\renewcommand\tabularxcolumn[1]{C{#1}}
%\usepackage[left=2cm,right=2cm,top=2cm,bottom=2cm]{geometry}
\usepackage{subcaption} 
\usepackage{caption}
\usepackage[left=2cm,right=2cm,top=1cm,bottom=2cm]{geometry}
%\usepackage{siunitx}



\begin{document}
\setlength{\parindent}{0cm} 



\frenchspacing

% Default margins are too wide all the way around. I reset them here
%\setlength{\topmargin}{-.5in}
%\setlength{\textheight}{9in}
%\setlength{\oddsidemargin}{.125in}
%\setlength{\evensidemargin}{.125in}
\setlength{\textwidth}{6.25in}

\title {\Large{\textbf{Mass Balance--CMFR with Reactive Pollutant}}\\ \large{CENG 340--Introduction to Environmental Engineering\\
Instructor: Deborah Sills\\ \textbf{In Class: September 25, 2013}}}

\author {}
\date {}
\maketitle

\vspace{-1.5cm}

A tanker truck overturns on the highway and leaks 4545 kg of wastewater from a yoghurt production facility into a small lake.  The wastewater is immediately and uniformly distributed throughout the lake.  The lake has a volume of V = 10,000 m$^3$.  A stream flows into and out of the lake at a flowrate of Q = 1000 $\mathrm{\frac{m^3}{day}}$.  Assume that bacteria present in the lake can degrade dairy waste with a first order rate coefficient of 0.005 day$^{-1}$. What is the concentration of the dairy waste in the lake as a function of time? Calculate how long it will take for the concentration of dairy waste in the lake to equal 5 percent of the initial concentration.


\section *{Step 1:} 
Draw a mass balance diagram, and label your diagram with given information and unknowns.

\vspace{0.2in}

\section *{Step 2:}
Write a general mass balance equation:
\vspace{0.2in}

\section *{Step 3:} 
Determine whether the system is at steady state or not.
\vspace{0.2in}

\section *{Step 4:}
Determine whether reactions occur or if conservative.
\vspace{0.2in}
\section *{Step 5:}
Rewrite the mass balance equation based on your answers in Step 3 and Step 4, and solve for C = f(t):




\end{document}