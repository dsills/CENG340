\documentclass[12pt,letterpaper]{article}
%\documentstyle[11pt]{article}
\usepackage[utf8]{inputenc}
\usepackage{amsmath}
\newcommand{\var}[1]{{\operatorname{#1}}}
%\usepackage{breqn}
\usepackage{xfrac}
\usepackage{amsfonts}
\usepackage{amssymb}
\usepackage[version = 3]{mhchem}
\usepackage{chemstyle}
%%For Table perhaps%%
%\usepackage{graphics}
\usepackage{graphicx}
\usepackage{epstopdf}
%\usepackage{tabularx,ragged2e,booktabs,caption}
%\newcolumntype{C}[1]{>{\Centering}m{#1}}
%\renewcommand\tabularxcolumn[1]{C{#1}}
\usepackage[left=2cm,right=2cm,top=0.5cm,bottom=2cm]{geometry}
\usepackage{subcaption} 
\usepackage{caption}
%\usepackage{siunitx}
%\usepackage{subfig}



\begin{document}
\setlength{\parindent}{0cm} 


\frenchspacing


% Default margins are too wide all the way around. I reset them here
%\setlength{\topmargin}{-.5in}
%\setlength{\textheight}{9in}
%\setlength{\oddsidemargin}{.125in}
%\setlength{\textwidth}{6.25in}




\title {\large{In Class Problems} \\ 
{\large \textbf{Precipitation--Dissolution Equilibrium for Tara}\\CENG 340--Introduction to Environmental Engineering\\
Instructor: Deborah Sills\\September 18, 2013}}

\author{}

\date {}
\maketitle
\vspace{-0.5in}

Tara asked a great question about the cadmium hydroxide precipitation problem we worked on in class on Monday. She wanted to know if we could use stoichiometry to calculate the equilibrium concentrations of [OH$^-$] and [Cd$^{2+}$] in the following equation, and if not, why?

\begin{align*}
\cee{[Cd(OH)_2]  &<=>[K_{sp}] [Cd^{2+}] + 2[OH^-]}
\end{align*}

where pK$_{sp}$ = 13.85\\

I told her that---within the context of Monday's problem---the answer was no, but that I would try to find a good way to explain why not.  I hope that the following problem will help:\\

If 50 mg of OH$^-$ and 50 mg of Cd$^{2+}$ are present in 1 L of water, what will be the final equilibrium concentration of Cd$^{2+}$? (Note that to solve this problem you need to use the equilibrium equation above.)
\begin{enumerate}

\item Take a look at the problem we solved on Monday, and describe how this problem is different from Monday's problem.  Write down how you think the differences between the two problems will change your approach to this one.


\item Try to solve the problem.\



\item Try to answer Tara's question.








\end{enumerate}
\end{document}