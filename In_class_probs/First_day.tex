\documentclass[12pt,letterpaper]{article}
%\documentstyle[11pt]{article}
\usepackage[utf8]{inputenc}
\usepackage{amsmath}
\usepackage{xfrac}
\usepackage{amsfonts}
\usepackage{amssymb}
\usepackage[version = 3]{mhchem}
\usepackage{chemstyle}
%%For Table perhaps%%
%\usepackage{graphics}
\usepackage{graphicx}
\usepackage{epstopdf}
%\usepackage{tabularx,ragged2e,booktabs,caption}
%\newcolumntype{C}[1]{>{\Centering}m{#1}}
%\renewcommand\tabularxcolumn[1]{C{#1}}
\usepackage[left=2cm,right=2cm,top=2cm,bottom=2cm]{geometry}
\usepackage{subcaption} 
\usepackage{caption}
\usepackage[colorlinks]{hyperref}
\usepackage[svgnames]{xcolor}
\hypersetup{citecolor=DeepPink4}
\hypersetup{linkcolor=DarkRed}
\hypersetup{urlcolor=DarkBlue}
\usepackage{cleveref}

\begin{document}
\setlength{\parindent}{0cm} 


\frenchspacing

\title {Gas Boom--First Day Problem} 
\author {CENG 340--Introduction to Environmental Engineering\\
Instructor: Prof. Deborah Sills}
%\date {, 2013}
\maketitle
In the last five years, there has been a gas boom in Pennsylvania. The boom has been driven by extensive gas reserves in the Marcellus Shale--a rock formation under several Middle Atlantic states and concentrated in Pennsylvania--combined with new technology. Technological advances that have allowed hydraulic fracturing to work in tandem with horizontal drilling have made extraction feasible. Hydraulic fracturing involves pumping a pressurized fluid through a wellbore which fractures bedrock and releases gas. Fracturing fluid is typically comprised of 98\% fresh water. The other 2\% is a mixture organic and inorganic chemicals that are added to prevent corrosion, change the hydrodynamic properties of the fluid and inhibit bacterial growth. Sand is also added to keep the fractures open.\\

Each new well is estimated to use 9 – 30 million liters of fracturing fluid. Upon release of pressure, an estimated 0.8 - 10 million liters of “flowback” water returns to the surface and requires appropriate treatment and disposal. The flowback contains most of the additives present in the original fluid, as well as brackish levels of dissolved salts, increased metal concentrations, as well as some radionuclides (e.g., barium). Though some of the flowback can be treated and reused onsite, a most of it will be shipped off to wastewater treatment facilities before being released into the watershed. One problem that flowback water causes at wastewater treatment facilities is due to the presence of surfactants (0.08\% w/v).  These compounds are added to reduce fluid tension and improve flow, but they cause unwanted foaming which blocks oxygen transfer during wastewater treatment.  As you will learn this semester, oxygen transfer is required for conventional wastewater treatment.\\

You have been hired to evaluate a biological treatment system that will reduce the concentration of the surfactant 2-butoxyethanol to 1 mg/L.\\

\textbf{ \emph{How would you begin to solve this problem?}}  What additional information do you need?  Describe potential problems that may arise with the proposed biological treatment system?  

\end{document}


