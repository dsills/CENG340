\documentclass[12pt,letterpaper]{article}
\usepackage[utf8]{inputenc}
\usepackage{amsmath}
\usepackage{amsfonts}
\usepackage{amssymb}
\usepackage[version = 3]{mhchem}
\usepackage{chemstyle}
%%For Table perhaps%%
%\usepackage{graphics}
\usepackage{graphicx}
\usepackage{epstopdf}
\usepackage[colorlinks]{hyperref}
\usepackage[svgnames]{xcolor}
\hypersetup{citecolor=DeepPink4}
\hypersetup{linkcolor=DarkRed}
\hypersetup{urlcolor=DarkBlue}
\usepackage{cleveref}
\usepackage{tabularx,ragged2e,booktabs,caption}
%\newcolumntype{C}[1]{>{\Centering}m{#1}}
\renewcommand\tabularxcolumn[1]{C{#1}}
\usepackage[left=2cm,right=2cm,top=2cm,bottom=2cm]{geometry}


\begin{document}
\setlength{\parindent}{0cm} 


\frenchspacing

\title{Response to Summer Reading \\ \vspace{2 mm} {\Large Blog Posts on the \emph{The Big Necessity} by Rose George}}

\author {CENG340---Introduction to Environmental Engineering\\ \vspace{2 mm} {Instructor: Prof. Deborah Sills}}
\date {}
\maketitle
%\vspace{-8 mm}

\section *{Due Dates (Blog posts due by 11:59 pm)}
\begin{enumerate}
\item First blog post: Friday 9/6.
\item First response to a classmate's post (on a chapter you did not blog on): Wednesday 9/11.
\item Second blog post: Wednesday 9/25.
\item Second response to a classmate's post (or a response to a response): Wednesday 10/2.
\item Third post: Friday 10/18.
\item Third response to  a classmate's post (or a response to a response): Wednesday 11/6.
\end{enumerate}

\section *{Learning Objectives}
\begin{enumerate}
\item Develop an appreciation of the impact of civil and environmental engineering on society and the environment.
\item Engage in an intellectual discussion about topics raised by the Rose George.
\item Practice communicating to the public about topics in civil and environmental engineering.
\end{enumerate}

\section *{Rationale}
In her book \emph{The Big Necessity---the Unmentionable World of Human Waste and Why It Matters}, Rose George covers a wide range of topics, all related to one of the central themes of this course: dealing with human waste. The topics covered include the miles of underground sewers in London, open defecation in the slums of India, and the ``most expensive toilet in the world,'' which was constructed for the U.S. Space Station.  I chose this book, because almost all of the chapters (except, perhaps, the one about the fancy Japanese toilets) remind me how civil and environmental engineers have improved our lives, and, on the other hand, the problems that existing engineering designs don't address. My hope is that our discussions of \emph{The Big Necessity}--- on a blog and in class---will help you appreciate ways that some of the topics we're covering this semester are applied in real life.

\section *{Assignment} This assignment and its associated rubric are based on \href{http://chronicle.com/blogs/profhacker/a-rubric-for-evaluating-student-blogs/27196}{Prof. Mark Sample's assignment}, described in the Chronicle of Higher Education.
\begin{enumerate}
\item Write three blog posts (\textbf{250--350 words each}) on three chapters of your choice from the \emph{The Big Necessity}. There are a couple of approaches you may take for this assignment: write about the reading within the context of civil and environmental engineering (discuss how engineering designs have improved human life, or write about designs that are still needed to address problems raised in the chapter); write about an aspect of the chapter that you don't understand, or something that surprised you; formulate a thought-provoking question or two about the reading and then try to answer your own questions. 

\item Write three responses (\textbf{100--200 words each}) to one of your classmates blog posts, or---for the second and third response assignments---you may respond to a response. Note that \textbf{you must respond in a respectful manner}. Make sure you draw on the text (i.e., \emph{The Big Necessity}), the post you are responding to, and, if you want, something we are doing in class, or other information you find online (please include a link).   You should try to build on your peer's post by disagreeing with it , rethinking it, and/or posing a question or two that came up for you while you read it. 
\end{enumerate}

Note that you may include links to news articles, videos, or other blogs if appropriate.



%\pagebreak

\section *{Grading}
The grade you earn from this assignment will make up 7\% of your overall course grade.  The blog posts (worth 20 pts each) and responses (worth 10 pts each) will be rated using the following rubric:\\\\

\begin{minipage}{\linewidth}
\centering
%\captionof{table}{Description of the model parameters} \label{tab:title}

\begin{tabular}{|c|p{14cm}|l|c|}\toprule[1.25pt]
\bf Rating	& \bf Characteristics\\\midrule
4	& \emph{Exceptional.} The post is clear, concise, and easy to follow. The writer integrates examples from the text with his own independent insights, and she considers alternate views when appropriate. The post demonstrates that the writer is deeply engaged with the topic.\\ \hline

3	& \emph{Good.} The post is reasonably clear, concise, and easy to follow. The writer's explanations are mostly based on examples or other evidence, and although independent insights are presented they are not fully developed. The post demonstrates that the writer is moderately engaged with the topic.\\ \hline

2	& \emph{Underdeveloped.} The blog post is mostly a description or a summary. The writer does not present new insights or alternate views, nor does she make connections between ideas. The post demonstrates that the writer is barely engaged with the topic.\\ \hline

1	& \emph{Limited.} The blog post is unfocused, difficult to follow, and demonstrates that the writer is not engaged with the topic.\\ \hline

0	& \emph{No Credit.} A blog post was not submitted, or it consists of one or two disconnected sentences.\\ \hline

\bottomrule[1.25pt]

\end {tabular}\par
\end{minipage}\\

\subsection *{Note:}
If you are unsure how to respond to this assignment, please come talk to me.  In addition, I strongly recommend that you consult with someone at the Writing Center.  They are very helpful.  I know because I consulted with them while as I developed this assignment.

\section *{Instructions for Getting Your Post onto the Wordpress site:}
\subsection *{Where is the blog?}
You should all be registered users of the blog.\\

\subsection *{Accessing the blog the first time:}
 Go to
 \href{http://blogs.bucknell.edu}{http://blogs.bucknell.edu} and select "Bucknell Blogs Login" on the right side of the site.  You should be directed to a page that allows you to access all of the  Bucknell Blogs that you are registered for, and you should be registered for our blog. 
 
\subsection *{Accessing the blog for the rest of the semester}
You should be able to access the blog directly:
\href{http://ceng340thebignecessity.blogs.bucknell.edu}{http://ceng340thebignecessity.blogs.bucknell.edu}.\\

Please let me know if you cannot access the blog, so I can contact my IT support person who will fix the problem. Make sure you check before the due date.

\subsection *{How to post}
Compose each blog post with your favorite word processing software (e.g., MS Word).  Do not compose the post online, because you may lose your work if WordPress crashes, which happens occasionally. Once you're happy with your post, go to the blog site:\\
 \href{http://ceng340thebignecessity.blogs.bucknell.edu}{http://ceng340thebignecessity.blogs.bucknell.edu}, and follow the posting directions shown on the next page.



\end{document}